\documentclass[a4paper,10pt]{article}
\usepackage[utf8]{inputenc}

%opening
\title{Foundations of Game Engine Development Volume 1 : Mathematics}
\author{Eric Lengyel}

\begin{document}

\maketitle

\begin{abstract}

    The following is a transcription of the book by Hanif Bin Ariffin.
    The engine of autism can power many things.

\end{abstract}

\tableofcontents

\section{Preface}

This book provides a detailed introduction to the mathematics used by modern game engine programmers.
The first three chapters cover the topics of linear algebera (vector and matrices), transforms, and geometry in a conventional manner common to many other textbooks on the subject.
This is done to provide a familiarity with the usual approach so that it's easy to make connections to similar expositions of the same topics elsewhere in the literature.
Along the way, we will make several attempts to foreshadow the discussion of a manifestly more elegant and more correct mathematical model that appears in the fourth cahpter.
The last quarter of the book endeavors to provide a deeper understanding of many of the concepts discussed eaerlir by introducing Grasmann algebra and geometric algebra.
Knowledge of these branches of mathematics also allows us to convey intuition and provide details in the first three chapters that are difficult to find elsewhere.

One of the goals in this book is to give practical engineering advice.
This is accomplished through many short code listings appearing throughout the book, showing how the mathematics we've discussed is implemented inside real-world game engines.
To avoid filling pages with code listings that are illustratively redundant, some data structures or functions referenced in various places have intentionally been left out.
This happens only when the form and behavior of the missing code is obvious.
For example, Chapter 1 includes code listings that define a data structure and operations corresponding to a three-dimensional vector, but it doesn't show similar code for a four dimensional vector, even though its used by other code listings later in the book, because it would be largely identical.
The complete library of code is available on the website cited below.

The book assume knowledge of C++, trigonometry and standard floating-point operations.
In Chapter 3, there will be a bit of calculus involved.
If you are already familiar with linear algebra, you could skip the first two chapters.
It is not adviseable to skip Chapter 3 as Chapter 4 requires familiarity with the former.

The source code for the listings provided in the book is available at : [TODO :: LINK]

Have fun and good luck.

\section{Vectors and Matrices}

Vector and matrices is one of the most fundamental building block in game engine development.
It is used practically everywhere.
Thus a strong grasp on this subject is required before going further.
With that in mind, this book will go through all the contents assuming only an existing proficiency in trigonometry on the part of the reader.

\subsection{Vector Fundamentals}

There are 2 ways to represent numerical quantities in geometry, physics and many other fields applied to virtual simulations.
One is a \textit{scalar} and a \textit{vector}.

A \textit{scalar} is a quantity such as distance, mass, or time that can be fully described by a single numerical value.
For instance, the number of apples is $X$, or the amount of health left is $Y$.

A \textit{vector} is a quantity that carries enough information to represent a direction in space in addition to a magnitude.
For example :

\begin{enumerate}
    \item
          A position in 3D space requires 3 numerical values.
    \item
          A projectile have both the direction (X,Y,Z) and a magnitude.
\end{enumerate}

66c\subsection{Basic Vector Operations}
\subsubsection{Magnitude and Scalar Operations}
\subsubsection{Addition and Subtraction}
\subsection{Matrix Fundamentals}
\subsection{Basic Matrix Operations}
\subsubsection{Addition, Subtraction, and Scalar Multiplication}
\subsubsection{Matric Multiplication}
\subsection{Vector Multiplication}
\subsubsection{Dot Product}
\subsubsection{Cross Product}
\subsubsection{Scalar Triple Project}
\subsection{Vector Projection}
\subsection{Matrix Inversion}
\subsubsection{Identity Matrices}
\subsubsection{Determinants}
\subsubsection{Elementary Matrices}
\subsubsection{Inverse Calculations}
\subsubsection{Inverses of Small Matrices}
\section{Transforms}
\subsection{Coordinate Spaces}
\subsubsection{Transformation Matrices}
\subsubsection{Orthogonal Transforms}
\subsubsection{Transform Composition}
\subsection{Rotations}
\subsubsection{Rotation About a Coordinate Axis}
\subsubsection{Rotation About an Arbitrary Axis}
\subsection{Reflections}
\subsection{Scales}
\subsection{Skews}
\subsection{Homogeneous Coordinates}
\subsection{Quatenions}
\subsubsection{Quaternion Fundamentals}
\subsubsection{Rotation With Quaternions}
\section{Geometry}
\section{Advanced Algebra}
\section{Index}

\end{document}
