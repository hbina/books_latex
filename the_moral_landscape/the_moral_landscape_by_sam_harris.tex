\documentclass[a4paper,12pt]{extbook}
\usepackage[utf8]{inputenc}

\title{The Moral Landscape : How Science Can Determine Human Values}
\author{Sam Harris}

\begin{document}

\maketitle
\tableofcontents

\newpage
\section{Introduction}

\subsection{The Moral Landscape}

The people of Albania have a venerable tradition of vendetta called \textit{Kanun} :
if a man commits a murder, his victim's family can kill any one of his male relatives in reprisal. 
If a boy has the misfortune of being the son or brother of a murderer, he must spend his days and nights in hiding, forgoing a proper education, adequate health care, and the pleasures of a normal life. 
Untold numbers of Albanian men and boys live as prisoners of their homes even now. 
Can we say that the Albanians are morally wrong to have structured their society in this way?
Is their tradition of blood feud a form of evil?
Are their values inferior to our own?

Most people imagine that science cannot pose, much less answer, questions of this sort. 
How could we ever say, as a matter of scientific fact, that one way of life is better, or moral, than another?
Whose defintion of ``better'' or ``moral'' would we use?
While many scientists now study the evolution of morality, as well as its underlying neurobiology, the purpose of their research is merely to describe how human beings think and behave. 
No one expects science to tell us how we \textit{ought} to think and behave. 
Controversies about human values are controversies about which science officially has no opinion. 

I will argue, however, that questions about values --- about meaning, morality, and life's larger purpose --- are really questions about the well-being of conscious creatures. 
Values, therefore, translate into facts that can be scientifically understood: regarding positive and negative social emotions, retributive impulses, the effects of specific laws and social institutions on human relationships, the neurophysiology of happiness and suffering, etc. 
The most important of these facts are bound to transcend culture --- just as facts about physical and mental health do. 
Cancer in the highlands of New Guinea is still cancer;
cholera is still cholea;
schizophrenia is still schizophrenia;
and so, too, I will argue, compassion is still compassion, and well-being is still well-being. 
And if these are important cultural differences in how people flourish --- if, for instance, there are incompatible but equivalent ways to raise happy, intelligent, and creative children --- these differences are also facts that must depend upon the organization of the human brain. 
In principle, therefore, we can account for the ways in which culture defines us within the context of neuroscience and psychology. 
The more we understand ourselves at the level of the brain, the more we will see that there are right and wrong answers to questions of human values. 

Of course, we will have to confront some ancient disagreements about the status of moral truth:
people who draw their worldview from religion generally believe that moral truths exists, but only because God has woven it into the very fabric of reality;
while those who lack such faith tend to think that notions of ``good'' and ``evil'' must be the products of evolutionary pressure and cultural invention. 
On the first account, to speak of ``moral truth'' is, of necessity, to invoke God;
on the second, it is merely to give voice to one's apish urges, cultural biases, and philosophical confusion. 
My purpose is to persuade you that both sides in this debate are wrong. 
The goal of this book is to begin a convesation about how moral truth can be understood in the context of science. 

While the argument I make in this book is bound to be controvversial, it rests on a very simple premise :
human well-being entirely depends in the events in the world and on states of human brain. 
Consequently, there must be scientific truths to be known about it. 
A more detailed understanding of these truths will force us to draw clear distinctions between different ways of living in society with one another, judging some to be better or worse, more or less true to the facts, and more or less ethical. 
Clearly, such insights could help us improve the quality of human life --- and this is where academic debate ends and choices affecting the lives of millions of people begin. 

I am not suggesting that we are guaranteed to resolve every moral controversy through science. 
Differences of opinion will remain --- but opinions will be increasingly constraints by facts. 
And it is important to realize that our inability to answer a question say nothing whther the question itself has an answer. 
Exactly how many people were bitten by mosquitoes in the last sixty seconds?
How many of these people will contract malaria?
How many will die as a result?
Given the technical challenges involved, no team of scientists could possibly respond to such questions. 
And yet we know that they admit of simple numerical answers. 
Does our inablity to gather the relevant data oblige us to respect all opinions equally?
Of course not,
In the same way, the fact that we not be able to resolve specific moral dilemmas does not suggest that all competing responses to them are equally valid. 
In my experience, mistaking \textit{no answers in practice} for \textit{fno answers in principle} is a great source of moral confusion. 

There are, for instance, twenty-one U. S. states that still allow corporal punishment in their schools. 
These are places where it is actually legal for a teacher to beat a child with a wooden board hard enough to raise large bruises and even to break the skin. 
Hundreds of thousands of children are subjected to this violence each year, almost exclusively in the South. 
Needless to say, the rationale for this behavior is explicitly religious :
for the Creator of the Universe Himself has told us not to spare the rod, lest we spoil the Child (Proverbs 13:24, 20:30 and 23:13-14). 
However, if we are actually concerned about human well-being, and would treat children in such a way as to promote it, we might wonder whether it is generally wise to subject little boys and girls to pain, terror, and public humiliation as a means of encouraging their cognitive and emotional development. 
Is there any doubt that this question \textit{has} an answer?
Is there any doubt that it matters that we get it right?
In fact, all the research indicates that corporal punishment is a disastrous practice, leading to more violence and social pathology --- and, perversely, to greater support for corporal punishment. 

But the deeper point is that there simply must be answers to questions of this kind, whether we know them or not. 
And these are not areas where we can afford to simply respect the ``traditions'' of others and agree to disagree. 
Why will science increasingly decide such questions?
Because the discrepant answers people give to them --- along with the consequences that follow in terms of human relationships, states of mind, acts of violence, entanglements with the law, etc. --- translate into differences in our brains, in the brains of others, and in the world at large. 
I hope to show that when talking about values, we are actually talking about an interdependant worlds of facts. 

There are facts to be understood about how thoughts and intentions arise in the human brain;
there are facts to be learned about how these mental states translate into behavior;
there are further facts to be known about how these behaviors influence the world and the experience of other conscious beings. 
We will see that facts of this sort exhaust what we can reasonably mean by terms like ``good'' and ``evil''. 
They will also increasingly fall within the purview of science and run far deeeper than a person's religious affiliation. 
Just as there is no such things as Christian physics or Muslim algebra, we will see that there is no such things as Christian or Muslim morality. 
Indeed, I will argue that morality should be considered an underdeveloped branch of science. 

Since the publication of my first book, \textit{The End of Faith}, I have had a privileged view of the ``culture wars'' --- both in the United States, between secular liberals and Christian conservatives, and in Europe, between largely irreligious societies and their growing Muslim populations. 
Having received tens of thousands of letters and emails from people at every pint on the continuum between faith and doubt, I can say with some confidence that a shared belief in the limitations of reason lies at the bottom these cultural divides. 
Both sides believe that reason is powerless to answer the most important questions in human life. 
And how a person perceives the gulf between facts and values seems to influence his views on almost every issue of social importance --- from the fighting of wars to the education of children. 

This rupture in our thinking has different consequences at each end of the political spectrum :
religious conversatives tend to believe that there are right answers to questions of meaning and morality, but only because the God of Abraham deems it so. 
They concede that ordinary facts can be discovered through rational inquiry, but they believe that values must come from a voice in a whirlwind. 
Scriptural literalism, intolerance of diversity, mistrust of science, disregard for the real causes of human and animal suffering --- too often, this is how the division between facts and values expresses itself on the religious right. 

Secular liberals, on the other hand, tend to imagine that no objectives answers to moral questions exist. 
While John Stuart Mill might conform to \textit{our} cultural ideal of goodness better than Osama Bin Laden does, most secularists suspect that Mill's ideas about right and wrong reach no closer to the Truth. 
Multiculturalism, moral relativism, political correctness, tolerance even of \textit{intolerance} --- these are familiar consequences of separating facts and values on the left. 

It should concern us that these two orientations are not equally empowering. 
Increasingly, secular democracies are left supine before the unreasoning zeal of old-time religion. 
The juxtaposition of conservative dogmatism and liberal doubt accounts for the decade that has been lost in the United States to a ban on federal funding for embryonic stem-cell research;
it explains the years of political distraction we have suffered, and will continue to suffer, over issues like abortion and gay marriage;
it lies at the bottom of current efforts to pass antiblasphemy laws at the United Nations (which would make it illegal for the citizens of member states to criticize religion);
it has hobbled the West in its generational war against radical Islam;
and it may yet refashion the societies of Europe into a new Caliphate. 
Knowing what the Creator of the Universe believes about right and wrong inspires religious conservatives to enforce this vision in the public sphere at almost any cost; not knowing what is right—or that anything can ever be truly right—often leads secular liberals to surrender their intellectual standards and political freedoms with both hands. 

The scientific community is predominantly secular and liberal --- and the concessions that scientists have made to religious dogmatism have been breathtaking. 
As we will see, the problem reaches as high as the National Academies of Science and the National Institutes of Health. 
Even the journal \textit{Nature}, the most influential scientific publication on Earth, has been unable to reliably police the boundary between reasoned discourse and pious fiction. 
I recently reviewed every appearance of the term ``religion'' in the journal going back ten years and found that \textit{Nature}’s editors have generally accepted Stephen J. Gould’s doomed notion of ``nonoverlapping magisteria'' --- the idea that science and religion, properly construed, cannot be in conflict because they constitute different domains of expertise. 
As one editorial put it, problems arise only when these disciplines ``stray onto each other’s territories and stir up trouble. ''
The underlying claim is that while science is the best authority on the workings of the physical universe, religion is the best authority on meaning, values, morality, and the good life. 
I hope to persuade you that this is not only untrue, it could not possibly be true. 
Meaning, values, morality, and the good life must relate to facts about the well-being of conscious creatures --- and, in our case, must lawfully depend upon events in the world and upon states of the human brain. 
Rational, open-ended, honest inquiry has always been the true source of insight into such processes. 
Faith, if it is ever right about anything, is right by accident. 

The scientific community’s reluctance to take a stand on moral issues has come at a price. 
It has made science appear divorced, in principle, from the most important questions of human life. 
From the point of view of popular culture, science often seems like little more than a hatchery for technology. 
While most educated people will concede that the scientific method has delivered centuries of fresh embarrassment to religion on matters of fact, it is now an article of almost unquestioned certainty, both inside and outside scientific circles, that science has nothing to say about what constitutes a good life. 
Religious thinkers in all faiths, and on both ends of the political spectrum, are united on precisely this point;
the defense one most often hears for belief in God is not that there is compelling evidence for His existence, but that faith in Him is the only reliable source of meaning and moral guidance. 
Mutually incompatible religious traditions now take refuge behind the same non sequitur. 

It seems inevitable, however, that science will gradually encompass life’s deepest questions --- and this is guaranteed to provoke a backlash. 
How we respond to the resulting collision of worldviews will influence the progress of science, of course, but it may also determine whether we succeed in building a global civilization based on shared values. 
The question of how human beings should live in the twenty-first century has many competing answers --- and most of them are surely wrong. 
Only a rational understanding of human well-being will allow billions of us to coexist peacefully, converging on the same social, political, economic, and environmental goals. 
A science of human flourishing may seem a long way off, but to achieve it, we must first acknowledge that the intellectual terrain actually exists. 

Throughout this book I make reference to a hypothetical space that I call ``the moral landscape'' --- a space of real and potential outcomes whose peaks correspond to the heights of potential well-being and whose valleys represent the deepest possible suffering. 
Different ways of thinking and behaving --- different cultural practices, ethical codes, modes of government, etc. --- will translate into movements across this landscape and, therefore, into different degrees of human flourishing. 
I’m not suggesting that we will necessarily discover one right answer to every moral question or a single best way for human beings to live. 
Some questions may admit of many answers, each more or less equivalent. 
However, the existence of multiple peaks on the moral landscape does not make them any less real or worthy of discovery. 
Nor would it make the difference between being on a peak and being stuck deep in a valley any less clear or consequential. 

To see that multiple answers to moral questions need not pose a problem for us, consider how we currently think about food :
no one would argue that there must be oneright food to eat. 
And yet there is still an objective difference between healthy food and poison. 
There are exceptions --- some people will die if they eat peanuts, for instance --- but we can account for these within the context of a rational discussion about chemistry, biology, and human health. 
The world’s profusion of foods never tempts us to say that there are no facts to be known about human nutrition or that all culinary styles must be equally healthy in principle. 

Movement across the moral landscape can be analyzed on many levels --- ranging from biochemistry to economics --- but where human beings are concerned, change will necessarily depend upon states and capacities of the human brain. 
While I fully support the notion of ``consilience'' in science --- and, therefore, view the boundaries between scientific specialties as primarily a function of university architecture and limitations on how much any one person can learn in a lifetime --- the primacy of neuroscience and the other sciences of mind on questions of human experience cannot be denied. 
Human experience shows every sign of being determined by, and realized in, states of the human brain. 

Many people seem to think that a universal conception of morality requires that we find moral principles that admit of no exceptions. 
If, for instance, it is truly wrong to lie, it must \textit{always} be wrong to lie --- and if one can find a single exception, any notion of moral truth must be abandoned. 
But the existence of moral truth --- that is, the connection between how we think and behave and our well-being --- does not require that we define morality in terms of unvarying moral precepts. 
Morality could be a lot like chess :
there are surely principles that generally apply, but they might admit of important exceptions. 
If you want to play good chess, a principle like ``Don’t lose your Queen'' is almost always worth following. 
But it admits of exceptions :
sometimes sacrificing your Queen is a brilliant thing to do;
occasionally, it is the \textit{only} thing you can do. 
It remains a fact, however, that from any position in a game of chess there will be a range of objectively good moves and objectively bad ones. 
If there are objective truths to be known about human well-being --- if kindness, for instance, is generally more conducive to happiness than cruelty is --- then science should one day be able to make very precise claims about which of our behaviors and uses of attention are morally good, which are neutral, and which are worth abandoning. 

While it is too early to say that we have a full understanding of how human beings flourish, a piecemeal account is emerging. 
Consider, for instance, the connection between early childhood experience, emotional bonding, and a person’s ability to form healthy relationships later in laife. 
We know, of course, that emotional neglect and abuse are not good for us, psychologically or socially. 
We also know that the effects of early childhood experience must be realized in the brain. 
Research on rodents suggests that parental care, social attachment, and stress regulation are governed, in part, by the hormones vasopressin and oxytocin, because they influence activity in the brain’s reward system. 
When asking why early childhood neglect is harmful to our psychological and social development, it seems reasonable to think that it might result from a disturbance in this same system. 

While it would be unethical to deprive young children of normal care for the purposes of experiment, society inadvertently performs such experiments every day. 
To study the effects of emotional deprivation in early childhood, one group of researchers measured the blood concentrations of oxytocin and vasopressin in two populations : 
children raised in traditional homes and children who spent their first years in an orphanage. 
As you might expect, children raised by the State generally do not receive normal levels of nurturing. 
They also tend to have social and emotional difficulties later in life. 
As predicted, these children failed to show a normal surge of oxytocin and vasopressin in response to physical contact with their adoptive mothers. 
The relevant neuroscience is in its infancy, but we know that our emotions, social interactions, and moral intuitions mutually influence one another. 
We grow attuned to our fellow human beings through these systems, creating culture in the process. 
Culture becomes a mechanism for further social, emotional, and moral development. 
There is simply no doubt that the human brain is the nexus of these influences. 
Cultural norms influence our thinking and behavior by altering the structure and function of our brains. 
Do you feel that sons are more desirable than daughters? 
Is obedience to parental authority more important than honest inquiry? 
Would you cease to love your child if you learned that he or she was gay? 
The ways parents view such questions, and the subsequent effects in the lives of their children, must translate into facts about their brains. 

My goal is to convince you that human knowledge and human values can no longer be kept apart. 
The world of measurement and the world of meaning must eventually be reconciled. 
And science and religion --- being antithetical ways of thinking about the same reality -- will never come to terms. 
As with all matters of fact, differences of opinion on moral questions merely reveal the incompleteness of our knowledge; 
they do not oblige us to respect a diversity of views indefinitely. 

\newpage
\section{Chapter 1 : Moral Truths}

\newpage
\section{Chapter 2 : Good And Evil}

\newpage
\section{Chapter 3 : Belief}

\newpage
\section{Chapter 4 : Religion}

\newpage
\section{Chapter 5 : The Future of Happiness}

\newpage
\section{Acknowledgements}

\newpage
\section{Notes}

\newpage
\section{References}

\newpage
\section{Index}

\end{document}
