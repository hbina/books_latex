\documentclass[a4paper,14pt]{extbook}
\usepackage[utf8]{inputenc}
\usepackage[margin=0.5in]{geometry}

\renewcommand{\baselinestretch}{1.5}

\title{The Moral Landscape : How Science Can Determine Human Values}
\author{Sam Harris}

\begin{document}

\maketitle
\tableofcontents

\newpage
\section{Introduction}

\subsection{The Moral Landscape}

The people of Albania have a venerable tradition of vendetta called \textit{Kanun} :
if a man commits a murder, his victim's family can kill any one of his male relatives in reprisal.
If a boy has the misfortune of being the son or brother of a murderer, he must spend his days and nights in hiding, forgoing a proper education, adequate health care, and the pleasures of a normal life.
Untold numbers of Albanian men and boys live as prisoners of their homes even now.
Can we say that the Albanians are morally wrong to have structured their society in this way?
Is their tradition of blood feud a form of evil?
Are their values inferior to our own?

Most people imagine that science cannot pose, much less answer, questions of this sort.
How could we ever say, as a matter of scientific fact, that one way of life is better, or moral, than another?
Whose definition of ``better'' or ``moral'' would we use?
While many scientists now study the evolution of morality, as well as its underlying neurobiology, the purpose of their research is merely to describe how human beings think and behave.
No one expects science to tell us how we \textit{ought} to think and behave.
Controversies about human values are controversies about which science officially has no opinion.

I will argue, however, that questions about values --- about meaning, morality, and life's larger purpose --- are really questions about the well-being of conscious creatures.
Values, therefore, translate into facts that can be scientifically understood: regarding positive and negative social emotions, retributive impulses, the effects of specific laws and social institutions on human relationships, the neurophysiology of happiness and suffering, etc.
The most important of these facts are bound to transcend culture --- just as facts about physical and mental health do.
Cancer in the highlands of New Guinea is still cancer;
cholera is still cholera;
schizophrenia is still schizophrenia;
and so, too, I will argue, compassion is still compassion, and well-being is still well-being.
And if these are important cultural differences in how people flourish --- if, for instance, there are incompatible but equivalent ways to raise happy, intelligent, and creative children --- these differences are also facts that must depend upon the organization of the human brain.
In principle, therefore, we can account for the ways in which culture defines us within the context of neuroscience and psychology.
The more we understand ourselves at the level of the brain, the more we will see that there are right and wrong answers to questions of human values.

Of course, we will have to confront some ancient disagreements about the status of moral truth:
people who draw their worldview from religion generally believe that moral truths exists, but only because God has woven it into the very fabric of reality;
while those who lack such faith tend to think that notions of ``good'' and ``evil'' must be the products of evolutionary pressure and cultural invention.
On the first account, to speak of ``moral truth'' is, of necessity, to invoke God;
on the second, it is merely to give voice to one's apish urges, cultural biases, and philosophical confusion.
My purpose is to persuade you that both sides in this debate are wrong.
The goal of this book is to begin a conversation about how moral truth can be understood in the context of science.

While the argument I make in this book is bound to be controversial, it rests on a very simple premise :
human well-being entirely depends in the events in the world and on states of human brain.
Consequently, there must be scientific truths to be known about it.
A more detailed understanding of these truths will force us to draw clear distinctions between different ways of living in society with one another, judging some to be better or worse, more or less true to the facts, and more or less ethical.
Clearly, such insights could help us improve the quality of human life --- and this is where academic debate ends and choices affecting the lives of millions of people begin.

I am not suggesting that we are guaranteed to resolve every moral controversy through science.
Differences of opinion will remain --- but opinions will be increasingly constraints by facts.
And it is important to realize that our inability to answer a question say nothing whether the question itself has an answer.
Exactly how many people were bitten by mosquitoes in the last sixty seconds?
How many of these people will contract malaria?
How many will die as a result?
Given the technical challenges involved, no team of scientists could possibly respond to such questions.
And yet we know that they admit of simple numerical answers.
Does our inability to gather the relevant data oblige us to respect all opinions equally?
Of course not,
In the same way, the fact that we not be able to resolve specific moral dilemmas does not suggest that all competing responses to them are equally valid.
In my experience, mistaking \textit{no answers in practice} for \textit{fno answers in principle} is a great source of moral confusion.

There are, for instance, twenty-one U. S. states that still allow corporal punishment in their schools.
These are places where it is actually legal for a teacher to beat a child with a wooden board hard enough to raise large bruises and even to break the skin.
Hundreds of thousands of children are subjected to this violence each year, almost exclusively in the South.
Needless to say, the rationale for this behavior is explicitly religious :
for the Creator of the Universe Himself has told us not to spare the rod, lest we spoil the Child (Proverbs 13:24, 20:30 and 23:13-14).
However, if we are actually concerned about human well-being, and would treat children in such a way as to promote it, we might wonder whether it is generally wise to subject little boys and girls to pain, terror, and public humiliation as a means of encouraging their cognitive and emotional development.
Is there any doubt that this question \textit{has} an answer?
Is there any doubt that it matters that we get it right?
In fact, all the research indicates that corporal punishment is a disastrous practice, leading to more violence and social pathology --- and, perversely, to greater support for corporal punishment.

But the deeper point is that there simply must be answers to questions of this kind, whether we know them or not.
And these are not areas where we can afford to simply respect the ``traditions'' of others and agree to disagree.
Why will science increasingly decide such questions?
Because the discrepant answers people give to them --- along with the consequences that follow in terms of human relationships, states of mind, acts of violence, entanglements with the law, etc. --- translate into differences in our brains, in the brains of others, and in the world at large.
I hope to show that when talking about values, we are actually talking about an interdependent worlds of facts.

There are facts to be understood about how thoughts and intentions arise in the human brain;
there are facts to be learned about how these mental states translate into behavior;
there are further facts to be known about how these behaviors influence the world and the experience of other conscious beings.
We will see that facts of this sort exhaust what we can reasonably mean by terms like ``good'' and ``evil''.
They will also increasingly fall within the purview of science and run far deeper than a person's religious affiliation.
Just as there is no such things as Christian physics or Muslim algebra, we will see that there is no such things as Christian or Muslim morality.
Indeed, I will argue that morality should be considered an underdeveloped branch of science.

Since the publication of my first book, \textit{The End of Faith}, I have had a privileged view of the ``culture wars'' --- both in the United States, between secular liberals and Christian conservatives, and in Europe, between largely irreligious societies and their growing Muslim populations.
Having received tens of thousands of letters and emails from people at every pint on the continuum between faith and doubt, I can say with some confidence that a shared belief in the limitations of reason lies at the bottom these cultural divides.
Both sides believe that reason is powerless to answer the most important questions in human life.
And how a person perceives the gulf between facts and values seems to influence his views on almost every issue of social importance --- from the fighting of wars to the education of children.

This rupture in our thinking has different consequences at each end of the political spectrum :
religious conservatives tend to believe that there are right answers to questions of meaning and morality, but only because the God of Abraham deems it so.
They concede that ordinary facts can be discovered through rational inquiry, but they believe that values must come from a voice in a whirlwind.
Scriptural literalism, intolerance of diversity, mistrust of science, disregard for the real causes of human and animal suffering --- too often, this is how the division between facts and values expresses itself on the religious right.

Secular liberals, on the other hand, tend to imagine that no objectives answers to moral questions exist.
While John Stuart Mill might conform to \textit{our} cultural ideal of goodness better than Osama Bin Laden does, most secularists suspect that Mill's ideas about right and wrong reach no closer to the Truth.
Multiculturalism, moral relativism, political correctness, tolerance even of \textit{intolerance} --- these are familiar consequences of separating facts and values on the left.

It should concern us that these two orientations are not equally empowering.
Increasingly, secular democracies are left supine before the unreasoning zeal of old-time religion.
The juxtaposition of conservative dogmatism and liberal doubt accounts for the decade that has been lost in the United States to a ban on federal funding for embryonic stem-cell research;
it explains the years of political distraction we have suffered, and will continue to suffer, over issues like abortion and gay marriage;
it lies at the bottom of current efforts to pass anti-blasphemy laws at the United Nations (which would make it illegal for the citizens of member states to criticize religion);
it has hobbled the West in its generational war against radical Islam;
and it may yet refashion the societies of Europe into a new Caliphate.
Knowing what the Creator of the Universe believes about right and wrong inspires religious conservatives to enforce this vision in the public sphere at almost any cost; not knowing what is right—or that anything can ever be truly right—often leads secular liberals to surrender their intellectual standards and political freedoms with both hands.

The scientific community is predominantly secular and liberal --- and the concessions that scientists have made to religious dogmatism have been breathtaking.
As we will see, the problem reaches as high as the National Academies of Science and the National Institutes of Health.
Even the journal \textit{Nature}, the most influential scientific publication on Earth, has been unable to reliably police the boundary between reasoned discourse and pious fiction.
I recently reviewed every appearance of the term ``religion'' in the journal going back ten years and found that \textit{Nature}’s editors have generally accepted Stephen J. Gould’s doomed notion of ``non-overlapping magisteria'' --- the idea that science and religion, properly construed, cannot be in conflict because they constitute different domains of expertise.
As one editorial put it, problems arise only when these disciplines ``stray onto each other’s territories and stir up trouble. ''
The underlying claim is that while science is the best authority on the workings of the physical universe, religion is the best authority on meaning, values, morality, and the good life.
I hope to persuade you that this is not only untrue, it could not possibly be true.
Meaning, values, morality, and the good life must relate to facts about the well-being of conscious creatures --- and, in our case, must lawfully depend upon events in the world and upon states of the human brain.
Rational, open-ended, honest inquiry has always been the true source of insight into such processes.
Faith, if it is ever right about anything, is right by accident.

The scientific community’s reluctance to take a stand on moral issues has come at a price.
It has made science appear divorced, in principle, from the most important questions of human life.
From the point of view of popular culture, science often seems like little more than a hatchery for technology.
While most educated people will concede that the scientific method has delivered centuries of fresh embarrassment to religion on matters of fact, it is now an article of almost unquestioned certainty, both inside and outside scientific circles, that science has nothing to say about what constitutes a good life.
Religious thinkers in all faiths, and on both ends of the political spectrum, are united on precisely this point;
the defense one most often hears for belief in God is not that there is compelling evidence for His existence, but that faith in Him is the only reliable source of meaning and moral guidance.
Mutually incompatible religious traditions now take refuge behind the same non sequitur.

It seems inevitable, however, that science will gradually encompass life’s deepest questions --- and this is guaranteed to provoke a backlash.
How we respond to the resulting collision of worldviews will influence the progress of science, of course, but it may also determine whether we succeed in building a global civilization based on shared values.
The question of how human beings should live in the twenty-first century has many competing answers --- and most of them are surely wrong.
Only a rational understanding of human well-being will allow billions of us to coexist peacefully, converging on the same social, political, economic, and environmental goals.
A science of human flourishing may seem a long way off, but to achieve it, we must first acknowledge that the intellectual terrain actually exists.

Throughout this book I make reference to a hypothetical space that I call ``the moral landscape'' --- a space of real and potential outcomes whose peaks correspond to the heights of potential well-being and whose valleys represent the deepest possible suffering.
Different ways of thinking and behaving --- different cultural practices, ethical codes, modes of government, etc. --- will translate into movements across this landscape and, therefore, into different degrees of human flourishing.
I’m not suggesting that we will necessarily discover one right answer to every moral question or a single best way for human beings to live.
Some questions may admit of many answers, each more or less equivalent.
However, the existence of multiple peaks on the moral landscape does not make them any less real or worthy of discovery.
Nor would it make the difference between being on a peak and being stuck deep in a valley any less clear or consequential.

To see that multiple answers to moral questions need not pose a problem for us, consider how we currently think about food :
no one would argue that there must be one right food to eat.
And yet there is still an objective difference between healthy food and poison.
There are exceptions --- some people will die if they eat peanuts, for instance --- but we can account for these within the context of a rational discussion about chemistry, biology, and human health.
The world’s profusion of foods never tempts us to say that there are no facts to be known about human nutrition or that all culinary styles must be equally healthy in principle.

Movement across the moral landscape can be analyzed on many levels --- ranging from biochemistry to economics --- but where human beings are concerned, change will necessarily depend upon states and capacities of the human brain.
While I fully support the notion of ``consilience'' in science --- and, therefore, view the boundaries between scientific specialties as primarily a function of university architecture and limitations on how much any one person can learn in a lifetime --- the primacy of neuroscience and the other sciences of mind on questions of human experience cannot be denied.
Human experience shows every sign of being determined by, and realized in, states of the human brain.

Many people seem to think that a universal conception of morality requires that we find moral principles that admit of no exceptions.
If, for instance, it is truly wrong to lie, it must \textit{always} be wrong to lie --- and if one can find a single exception, any notion of moral truth must be abandoned.
But the existence of moral truth --- that is, the connection between how we think and behave and our well-being --- does not require that we define morality in terms of unvarying moral precepts.
Morality could be a lot like chess :
there are surely principles that generally apply, but they might admit of important exceptions.
If you want to play good chess, a principle like ``Don’t lose your Queen'' is almost always worth following.
But it admits of exceptions :
sometimes sacrificing your Queen is a brilliant thing to do;
occasionally, it is the \textit{only} thing you can do.
It remains a fact, however, that from any position in a game of chess there will be a range of objectively good moves and objectively bad ones.
If there are objective truths to be known about human well-being --- if kindness, for instance, is generally more conducive to happiness than cruelty is --- then science should one day be able to make very precise claims about which of our behaviors and uses of attention are morally good, which are neutral, and which are worth abandoning.

While it is too early to say that we have a full understanding of how human beings flourish, a piecemeal account is emerging.
Consider, for instance, the connection between early childhood experience, emotional bonding, and a person’s ability to form healthy relationships later in life.
We know, of course, that emotional neglect and abuse are not good for us, psychologically or socially.
We also know that the effects of early childhood experience must be realized in the brain.
Research on rodents suggests that parental care, social attachment, and stress regulation are governed, in part, by the hormones vasopressin and oxytocin, because they influence activity in the brain’s reward system.
When asking why early childhood neglect is harmful to our psychological and social development, it seems reasonable to think that it might result from a disturbance in this same system.

While it would be unethical to deprive young children of normal care for the purposes of experiment, society inadvertently performs such experiments every day.
To study the effects of emotional deprivation in early childhood, one group of researchers measured the blood concentrations of oxytocin and vasopressin in two populations :
children raised in traditional homes and children who spent their first years in an orphanage.
As you might expect, children raised by the State generally do not receive normal levels of nurturing.
They also tend to have social and emotional difficulties later in life.
As predicted, these children failed to show a normal surge of oxytocin and vasopressin in response to physical contact with their adoptive mothers.
The relevant neuroscience is in its infancy, but we know that our emotions, social interactions, and moral intuitions mutually influence one another.
We grow attuned to our fellow human beings through these systems, creating culture in the process.
Culture becomes a mechanism for further social, emotional, and moral development.
There is simply no doubt that the human brain is the nexus of these influences.
Cultural norms influence our thinking and behavior by altering the structure and function of our brains.
Do you feel that sons are more desirable than daughters?
Is obedience to parental authority more important than honest inquiry?
Would you cease to love your child if you learned that he or she was gay?
The ways parents view such questions, and the subsequent effects in the lives of their children, must translate into facts about their brains.

My goal is to convince you that human knowledge and human values can no longer be kept apart.
The world of measurement and the world of meaning must eventually be reconciled.
And science and religion --- being antithetical ways of thinking about the same reality -- will never come to terms.
As with all matters of fact, differences of opinion on moral questions merely reveal the incompleteness of our knowledge;
they do not oblige us to respect a diversity of views indefinitely.

\subsection{Facts and Values}

The eighteenth-century Scottish philosopher David Hume famously argued that no description of the way the world is (facts) can tell us how we ought to behave (morality).
Following Hume, the philosopher G. E. Moore declared that any attempt to locate moral truths in the natural world was to commit a ``naturalistic fallacy.''
Moore argued that goodness could not be equated with any property of human experience (e.g., pleasure, happiness, evolutionary fitness) because it would always be appropriate to ask whether the property on offer was itself \textit{good}.
If, for instance, we were to say that goodness is synonymous with whatever gives people pleasure, it would still be possible to worry whether a specific instance of pleasure is actually \textit{good}.
This is known as Moore’s ``open question argument.''
And while I think this verbal trap is easily avoided when we focus on human well-being, most scientists and public intellectuals appear to have fallen into it.
Other influential philosophers, including Karl Popper, have echoed Hume and Moore on this point, and the effect has been to create a firewall between facts and values throughout our intellectual discourse.

While psychologists and neuroscientists now routinely study human happiness, positive emotions, and moral reasoning, they rarely draw conclusions about how human beings ought to think or behave in light of their findings.
In fact, it seems to be generally considered intellectually disreputable, even vaguely authoritarian, for a scientist to suggest that his or her work offers some guidance about how people should live.
The philosopher and psychologist Jerry Fodor crystallizes the view :

Science is about facts, not norms;
it might tell us how we are, but it couldn’t tell us what is wrong with how we are.
There couldn’t be a science of the human condition.

While it is rarely stated this clearly, this faith in the intrinsic limits of reason is now the received opinion in intellectual circles.

Despite the reticence of most scientists on the subject of good and evil, the scientific study of morality and human happiness is well underway.
This research is bound to bring science into conflict with religious orthodoxy and popular opinion --- just as our growing understanding of evolution has --- because the divide between facts and values is illusory in at least three senses :

\begin{enumerate}

    \item
          Whatever can be known about maximizing the well-being of conscious creatures --- which is, I will argue, the only thing we can reasonably value --- must at some point translate into facts about brains and their interaction with the world at large;
    \item
          The very idea of ``objective'' knowledge (i.e., knowledge acquired through honest and reasoning) has values built into it, as every effort we make to discuss facts depends upon principles that we must first value (i.e. logical consistency, reliance on evidence, parsimony, etc.);
    \item
          Beliefs about facts and beliefs about values seem to arise from similar processes at the level of the brain:
          it appears that we have a common system for judging truth and falsity in both domains.

\end{enumerate}


I will discuss each of these points in greater detail below.
Both in terms of what there is to know about the world and the brain mechanisms that allow us to know it, we will see that a clear boundary between facts and values simply does not exist.

Many readers might wonder how can we base our values on something as difficult to define as ``well-being''?
It seems to me, however, that the concept of well-being is like the concept of physical health :
it resists precise definition, and yet it is indispensable.
In fact, the meanings of both terms seem likely to remain perpetually open to revision as we make progress in science.
Today, a person can consider himself physically healthy if he is free of detectable disease, able to exercise, and destined to live into his eighties without suffering obvious decrepitude.
But this standard may change.
If the biogerontologist Aubrey de Grey is correct in viewing aging as an engineering problem that admits of a full solution, being able to walk a mile on your hundredth birthday will not always constitute ``health.''
There may come a time when not being able to run a marathon at age five hundred will be considered a profound disability.
Such a radical transformation of our view of human health would not suggest that current notions of health and sickness are arbitrary, merely subjective, or culturally constructed.
Indeed, the difference between a healthy person and a dead one is about as clear and consequential a distinction as we ever make in science.
The differences between the heights of human fulfillment and the depths of human misery are no less clear, even if new frontiers await us in both directions.

If we define ``good'' as that which supports well-being, as I will argue we must, the regress initiated by Moore’s ``open question argument'' really does stop.
While I agree with Moore that it is reasonable to wonder whether maximizing pleasure in any given instance is ``good,'' it makes no sense at all to ask whether maximizing well-being is ``good.''
It seems clear that what we are really asking when we wonder whether a certain state of pleasure is ``good,'' is whether it is conducive to, or obstructive of, some deeper form of well-being.
This question is perfectly coherent;
it surely has an answer (whether or not we are in a position to answer it);
and yet, it keeps notions of goodness anchored to the experience of sentient beings.

Defining goodness in this way does not resolve all questions of value;
it merely directs our attention to what values actually are --- the set of attitudes, choices, and behaviors that potentially affect our well-being, as well as that of other conscious minds.
While this leaves the question of what constitutes well-being genuinely open, there is every reason to think that this question has a finite range of answers.
Given that change in the well-being of conscious creatures is bound to be a product of natural laws, we must expect that this space of possibilities --- the moral landscape --- will increasingly be illuminated by science.

It is important to emphasize that a scientific account of human values --- i.e., one that places them squarely within the web of influences that link states of the world and states of the human brain --- is not the same as an evolutionary account.
Most of what constitutes human well-being at this moment escapes any narrow Darwinian calculus.
While the possibilities of human experience must be realized in the brains that evolution has built for us, our brains were not designed with a view to our ultimate fulfillment.
Evolution could never have foreseen the wisdom or necessity of creating stable democracies, mitigating climate change, saving other species from extinction, containing the spread of nuclear weapons, or of doing much else that is now crucial to our happiness in this century.

As the psychologist Steven Pinker has observed, if conforming to the dictates of evolution were the foundation of subjective well-being, most men would discover no higher calling in life than to make daily contributions to their local sperm bank.
After all, from the perspective of a man’s genes, there could be nothing more fulfilling than spawning thousands of children without incurring any associated costs or responsibilities.
But our minds do not merely conform to the logic of natural selection.
In fact, anyone who wears eyeglasses or uses sunscreen has confessed his disinclination to live the life that his genes have made for him.
While we have inherited a multitude of yearnings that probably helped our ancestors survive and reproduce in small bands of hunter-gatherers, much of our inner life is frankly incompatible with our finding happiness in today’s world.
The temptation to start each day with several glazed donuts and to end it with an extramarital affair might be difficult for some people to resist, for reasons that are easily understood in evolutionary terms, but there are surely better ways to maximize one’s long-term well-being.
I hope it is clear that the view of ``good'' and ``bad'' I am advocating, while fully constrained by our current biology (as well as by its future possibilities), cannot be directly reduced to instinctual drives and evolutionary imperatives.
As with mathematics, science, art, and almost everything else that interests us, our modern concerns about meaning and morality have flown the perch built by evolution.

\subsection{The Importance of Belief}

The human brain is an engine of belief.
Our minds continually consume, produce, and attempt to integrate ideas about ourselves and the world that purport to be true :
\textit{Iran is developing nuclear weapons;}
\textit{the seasonal flu can be spread through casual contact;}
\textit{I actually look better with gray hair.}
What must we do to believe such statements?
What, in other words, must a brain do to accept such propositions as true?
This question marks the intersection of many fields :
psychology, neuroscience, philosophy, economics, political science, and even jurisprudence.

Belief also bridges the gap between facts and values.
We form beliefs about facts :
and belief in this sense constitutes most of what we know about the world --- through science, history, journalism, etc.
But we also form beliefs about values :
judgments about morality, meaning, personal goals, and life’s larger purpose.
While they might differ in certain respects, beliefs in these two domains share very important features.
Both types of belief make tacit claims about right and wrong :
claims not merely about how we think and behave, but about how we should think and behave.
Factual beliefs like ``water is two parts hydrogen and one part oxygen'' and ethical beliefs like ``cruelty is wrong'' are not expressions of mere preference.
To really believe either proposition is also to believe that you have accepted it for legitimate reasons.
It is, therefore, to believe that you are in compliance with certain norms --- that you are sane, rational, not lying to yourself, not confused, not overly biased, etc.
When we believe that something is factually true or morally good, we also believe that another person, similarly placed, should share our belief.
This seems unlikely to change.
In chapter 3, we will see that both the logical and neurological properties of belief further suggest that the divide between facts and values is illusory.

\subsection{The Bad Life and the Good Life}

For my argument about the moral landscape to hold, I think one need only grant two points :

\begin{enumerate}
    \item Some people have better lives than others, and
    \item These differences relate, in some lawful and not entirely arbitrary ways, to states of the human brain and to states of the world.
\end{enumerate}

To make these premises less abstract, consider two generic lives that lie somewhere near the extremes on this continuum :

\subsubsection{The Bad Life}

You are a young widow who has lived her entire life in the midst of civil war.
Today, your seven-year-old daughter was raped and dismembered before your eyes.
Worse still, the perpetrator was your fourteen-year-old son, who was goaded to this evil at the point of a machete by a press gang of drug-addicted soldiers.
You are now running barefoot through the jungle with killers in pursuit.
While this is the worst day of your life, it is not entirely out of character with the other days of your life :
since the moment you were born, your world has been a theater of cruelty and violence.
You have never learned to read, taken a hot shower, or traveled beyond the green hell of the jungle.
Even the luckiest people you have known have experienced little more than an occasional respite from chronic hunger, fear, apathy, and confusion.
Unfortunately, you've been very unlucky, even by these bleak standards.
Your life has been one long emergency, now it is nearly over.

\subsubsection{The Good Life}

You are married to the most loving, intelligent, and charismatic person you have ever met.
Both of you have careers that are intellectually stimulating, and financially rewarding.
For decades, your wealth and social connections have allowed you to devote yourself to activities that bring you immense personal satisfaction.
One of your greatest sources of happiness has been to find creative ways to help people who have not had your good fortune in life.
In fact, you have just won a billion-dollar grant to benefit children in the developing world.
If asked, you would say that you could not imagine how your time on Earth could be better spent.
Due to a combination of good genes and optimal circumstances, you and your closest friends and family will live very long, healthy lives, untouched by crimes, sudden bereavements, and other misfortunes.

The examples I have picked, while generic, are nonetheless real --- in that they represent lives that some human beings are likely to be leading at this moment.
While there are surely ways in which this spectrum of suffering and happiness might be extended, I think these cases indicate the general range of experience that is accessible, in principle, to most of us.
I also think it is indisputable that most of what we do with our lives is predicated on there being nothing more important, at least for ourselves and for those closest to us, than the difference between the Bad Life and the Good Life.

Let me simply concede that if you don't see a distinction between these two lives that is worth valuing (premise 1 above), there may be nothing I can say that will attract you to my view of the moral landscape.
Likewise, if you admit that these lives are different, and that one is surely better than the other, but you believe these differences have no lawful relationship to human behavior, societal conditions, or states of the brain (premise 2), then you will also fail to see the point of my argument.
While I don't see how either premise 1 or 2 can be reasonably doubted, my experience discussing these issues suggests that I should address such skepticism, however far-fetched it may seem.

There are actually people who claim to be unimpressed by the difference between the Bad Life and the Good Life.
I have even met people who will go so far as to deny that any difference exists.
While they will acknowledge that we habitually speak and act as if there were a continuum of experience that can be described by words like ``misery,'' ``terror,'' ``agony,'' ``madness,'' etc., on one end and ``well-being,'' ``happiness,'' ``peace,'' ``bliss,'' etc., on the other, when the conversation turns to philosophical and scientific matters, such people will say learned things like, ``but, of course, that is just how we play our particular language game. It doesn’t mean there is a difference in reality.''
One hopes that these people take life’s difficulties in stride.
They also use words like ``love'' and ``happiness,'' from time to time, but we should wonder what these terms could signify that does not entail a preference for the Good Life over the Bad Life.
Anyone who claims to see no difference between these two states of being (and their concomitant worlds), should be just as likely to consign himself and those he ``loves'' to one or the other at random and call the result ``happiness.''

Ask yourself, if the difference between the Bad Life and the Good Life doesn’t matter to a person, what could possibly matter to him?
Is it conceivable that something might matter more than this difference, expressed on the widest possible scale?
What would we think of a person who said, ``Well, I could have delivered all seven billion of us into the Good Life, but I had other priorities.''
Would it be \textit{possible} to have other priorities?
Wouldn’t any real priority be best served amid the freedom and opportunity afforded by the Good Life?
Even if you happen to be a masochist who fancies an occasional taunting with a machete, wouldn’t this desire be best satisfied in the context of the Good Life?

Imagine someone who spends all his energy trying to move as many people as possible toward the Bad Life, while another person is equally committed to undoing this damage and moving people in the opposite direction:
Is it conceivable that you or anyone you know could overlook the differences between these two projects?
Is there any possibility of confusing them or their underlying motivations?
And won’t there necessarily be objective conditions for these differences?
If, for instance, one’s goal were to place a whole population securely in the Good Life, wouldn’t there be more and less effective ways of doing this?
How would forcing boys to rape and murder their female relatives fit into the picture?

I do not mean to belabor the point, but the point is crucial --- and there is a pervasive assumption among educated people that either such differences don’t exist, or that they are too variable, complex, or culturally idiosyncratic to admit of general value judgments.
However, the moment one grants there is a difference between the Bad Life and the Good Life that lawfully relates to states of the human brain, to human behavior, and to states of the world, one has admitted that there are right and wrong answers to questions of morality.
To make sure this point is nailed down, permit me to consider a few more objections :

\textit{What if, seen in some larger context, the Bad Life is actually better than the Good Life --- e.g., what if all those child soldiers will be happier in some afterlife, because they have been purified of sin or have learned to call God by the right name, while the people in the Good Life will get tortured in some physical hell for eternity?}

If the universe is really organized this way, much of what I believe will stand corrected on the Day of Judgment.
However, my basic claim about the connection between facts and values would remain unchallenged.
The rewards and punishments of an afterlife would simply alter the temporal characteristics of the moral landscape.
If the Bad Life is actually better over the long run than the Good Life --- because it wins you endless happiness, while the Good Life represents a mere dollop of pleasure presaging an eternity of suffering --- then the Bad Life would surely be better than the Good Life.
If this were the way the universe worked, we would be morally obligated to engineer an appropriately pious Bad Life for as many people as possible.
Under such a scheme, there would still be right and wrong answers to questions of morality, and these would still be assessed according to the experience of conscious beings.
The only thing left to be decided is how reasonable it is to worry that the universe might be structured in so bizarre a way.
It is not reasonable at all, I think --- but that is a different discussion.

\textit{What if certain people would actually prefer the Bad Life to the Good Life?
    Perhaps there are psychopaths and sadists who can expect to thrive in the context of the Bad Life and would enjoy nothing more than killing other people with machetes.}

Worries like this merely raise the question of how we should value dissenting opinions.
Jeffrey Dahmer’s idea of a life well lived was to kill young men, have sex with their corpses, dismember them, and keep their body parts as souvenirs.
We will confront the problem of psychopathy in greater detail in chapter 3.
For the moment, it seems sufficient to notice that in any domain of knowledge, we are free to say that certain opinions do not count.
In fact, we must say this for knowledge or expertise to count at all.
Why should it be any different on the subject of human well-being?

Anyone who doesn’t see that the Good Life is preferable to the Bad Life is unlikely to have anything to contribute to a discussion about human well-being.
Must we really argue that beneficence, trust, creativity, etc., enjoyed in the context of a prosperous civil society are better than the horrors of civil war endured in a steaming jungle filled with aggressive insects carrying dangerous pathogens?
I don’t think so.
In the next chapter, I will argue that anyone who would seriously maintain that the opposite is the case --- or even that it \textit{might} be the case --- is either misusing words or not taking the time to consider the details.

If we were to discover a new tribe in the Amazon tomorrow, there is not a scientist alive who would assume \textit{a priori} that these people must enjoy optimal physical health and material prosperity.
Rather, we would ask questions about this tribe’s average lifespan, daily calorie intake, the percentage of women dying in childbirth, the prevalence of infectious disease, the presence of material culture, etc.
Such questions would have answers, and they would likely reveal that life in the Stone Age entails a few compromises.
And yet news that these jolly people enjoy sacrificing their firstborn children to imaginary gods would prompt many (even most) anthropologists to say that this tribe was in possession of an alternate moral code every bit as valid and impervious to refutation as our own.
However, the moment one draws the link between morality and well-being, one sees that this is tantamount to saying that the members of this tribe must be as fulfilled, psychologically and socially, as any people on earth.
The disparity between how we think about physical health and mental/societal health reveals a bizarre double standard :
one that is predicated on our not knowing --- or, rather, on our \textit{pretending} not to know --- anything at all about human well-being.

Of course, some anthropologists have refused to follow their colleagues over the cliff.
Robert Edgerton performed a book-length exorcism on the myth of the ``noble savage,'' detailing the ways in which the most influential anthropologists of the 1920s and 1930s --- such as Franz Boas, Margaret Mead, and Ruth Benedict --- systematically exaggerated the harmony of folk societies and ignored their all too frequent barbarism or reflexively attributed it to the malign influence of colonialists, traders, missionaries, and the like.
Edgerton details how this romance with mere difference set the course for the entire field.
Thereafter, to compare societies in moral terms was deemed impossible.
Rather, it was believed that one could only hope to understand and accept a culture on its own terms.
Such cultural relativism became so entrenched that by 1939 one prominent Harvard anthropologist wrote that this suspension of judgment was ``probably the most meaningful contribution which anthropological studies have made to general knowledge.''
Let’s hope not.
In any case, it is a contribution from which we are still struggling to awaken.

Many social scientists incorrectly believe that all long-standing human practices must be evolutionarily adaptive :
for how else could they persist?
Thus, even the most bizarre and unproductive behaviors --- female genital excision, blood feuds, infanticide, the torture of animals, scarification, foot binding, cannibalism, ceremonial rape, human sacrifice, dangerous male initiations, restricting the diet of pregnant and lactating mothers, slavery, potlatch, the killing of the elderly, sati, irrational dietary and agricultural taboos attended by chronic hunger and malnourishment, the use of heavy metals to treat illness, etc. --- have been rationalized, or even idealized, in the fire-lit scribblings of one or another dazzled ethnographer.
But the mere endurance of a belief system or custom does not suggest that it is adaptive, much less wise.
It merely suggests that it hasn’t led directly to a society’s collapse or killed its practitioners outright.

The obvious difference between genes and \textit{memes} (e.g., beliefs, ideas, cultural practices) is also important to keep in view.
The latter are \textit{communicated};
they do not travel with the gametes of their human hosts.
The survival of memes, therefore, is not dependent on their conferring some actual benefit (reproductive or otherwise) on individuals or groups.
It is quite possible for people to traffic in ideas and other cultural products that diminish their well-being for centuries on end.

Clearly, people can adopt a form of life that needlessly undermines their physical health --- as the average lifespan in many primitive societies is scarcely a third of what it has been in the developed world since the middle of the twentieth century.
Why isn’t it equally obvious that an ignorant and isolated people might undermine their psychological well-being or that their social institutions could become engines of pointless cruelty, despair, and superstition?
Why is it even slightly controversial to imagine that some tribe or society could harbor beliefs about reality that are not only false but demonstrably harmful?

Every society that has ever existed has had to channel and subdue certain aspects of human nature --- envy, territorial violence, avarice, deceit, laziness, cheating, etc. --- through social mechanisms and institutions.
It would be a miracle if all societies --- irrespective of size, geographical location, their place in history, or the genomes of their members --- had done this equally well.
And yet the prevailing bias of cultural relativism assumes that such a miracle has occurred not just once, but always.
Let’s take a moment to get our bearings.
From a factual point of view, is it possible for a person to believe the wrong things?
Yes.
It is possible for a person to value the wrong things (that is, to believe the wrong things about human well-being)?
I am arguing that the answer to this question is an equally emphatic ``yes'' and, therefore, that science should increasingly inform our values.
Is it possible that certain people are incapable of wanting what they should want?
Of course --- just as there will always be people who are unable to grasp specific facts or believe certain true propositions.
As with every other description of a mental capacity or incapacity, these are ultimately statements about the human brain.

\subsection{Can Suffering Be Good?}

It seems clear that ascending the slopes of the moral landscape may sometimes require suffering.
It may also require negative social emotions, like guild and indignation.
Again, the analogy with physical health seems useful :
we must occasionally experience some unpleasantness --- medication, surgery, etc. --- in order to avoid greater suffering or death.
This principle seems to apply throughout our lives.
Merely learning to read or to play a new sport can produce feelings of deep frustration.
And yet there is little question that acquiring such skills generally improves our lives.
Even periods of depression may lead to better life decisions and to creative insights.
This seems to be the way our minds work.
So be it.

Of course, this principle also applies to civilization as a whole.
Merely making necessary improvements to a city’s infrastructure greatly inconveniences millions of people.
And unintended effects are always possible.
For instance, the most dangerous road on earth now appears to be a two-lane highway between Kabul and Jalalabad.
When it was unpaved, cratered, and strewn with boulders, it was comparatively safe.
But once some helpful Western contractors improved it, the driving skills of the local Afghans were finally liberated from the laws of physics.
Many now have a habit of passing slow-moving trucks on blind curves, only to find themselves suddenly granted a lethally unimpeded view of a thousand-foot gorge.
Are there lessons to be learned from such missteps in the name of progress?
Of course.
But they do not negate the reality of progress.
Again, the difference between the Good Life and the Bad Life could not be clearer :
the question, for both individuals and groups, is how can we most reliably move in one direction and avoid moving in the other?

\subsubsection{The Problem of Religion}

Anyone who wants to understand the world should be open to new facts and new arguments, even on subjects where his or her views are very well established.
Similarly, anyone truly interested in morality --- in the principles of behavior that allow people to flourish --- should be open to new evidence and new arguments that bear upon questions of happiness and suffering.
Clearly, the chief enemy of open conversation is dogmatism in all its forms.
Dogmatism is a well-recognized obstacle to scientific reasoning;
and yet, because scientists have been reluctant even to imagine that they might have something prescriptive to say about values, dogmatism is still granted remarkable scope on questions of both truth and goodness under the banner of religion.

In the fall of 2006, I participated in a three-day conference at the Salk Institute entitled Beyond Belief : Science, Religion, Reason, and Survival.
This event was organized by Roger Bingham and conducted as a town-hall meeting before an audience of invited guests.
Speakers included Steven Weinberg, Harold Kroto, Richard Dawkins, and many other scientists and philosophers who have been, and remain, energetic opponents of religious dogmatism and superstition.
It was a room full of highly intelligent, scientifically literate people --- molecular biologists, anthropologists, physicists, and engineers --- and yet, to my amazement, three days were insufficient to force agreement on the simple question of whether there is any conflict at all between religion and science.
Imagine a meeting of mountaineers unable to agree about whether their sport ever entails walking uphill, and you will get a sense of how bizarre our deliberations began to seem.

While at Salk, I witnessed scientists giving voice to some of the most dishonest religious apologies I have ever heard.
It is one thing to be told that the pope is a peerless champion of reason and that his opposition to embryonic stem-cell research is both morally principled and completely uncontaminated by religious dogmatism;
it is quite another to be told this by a Stanford physician who sits on the President’s Council on Bioethics.
Over the course of the conference, I had the pleasure of hearing that Hitler, Stalin, and Mao were examples of secular reason run amok, that the Islamic doctrines of martyrdom and jihad are not the cause of Islamic terrorism, that people can never be argued out of their beliefs because we live in an irrational world, that science has made no important contributions to our ethical lives (and cannot), and that it is not the job of scientists to undermine ancient mythologies and, thereby, “take away people’s hope” --- all from atheist scientists who, while insisting on their own skeptical hardheadedness, were equally adamant that there was something feckless and foolhardy, even indecent, about criticizing religious belief.
There were several moments during our panel discussions that brought to mind the final scene of Invasion of the Body Snatchers :
people who looked like scientists, had published as scientists, and would soon be returning to their labs, nevertheless gave voice to the alien hiss of religious obscurantism at the slightest prodding.
I had previously imagined that the front lines in our culture wars were to be found at the entrance to a megachurch.
I now realized that we have considerable work to do in a nearer trench.

I have made the case elsewhere that religion and science are in a zero-sum conflict with respect to facts.
Here, I have begun to argue that the division between facts and values is intellectually unsustainable, especially from the perspective of neuroscience.
Consequently, it should come as no surprise that I see very little room for compromise between faith and reason on questions of morality.
While religion is not the primary focus of this book, any discussion about the relationship between facts and values, the nature of belief, and the role of science in public discourse must continually labor under the burden of religious opinion.
I will, therefore, examine the conflict between religion and science in greater depth in chapter 4.

But there is no mystery why many scientists feel that they must pretend that religion and science are compatible.
We have recently emerged --- some of us leaping, some shuffling, others crawling --- out of many dark centuries of religious bewilderment and persecution, into an age when mainstream science is still occasionally treated with overt hostility by the general public and even by governments.
While few scientists living in the West now fear torture or death at the hands of religious fanatics, many will voice concerns about losing their funding if they give offense to religion, particularly in the United States.
It also seems that, given the relative poverty of science, wealthy organizations like the Templeton Foundation (whose endowment currently stands at \$1.5 billion) have managed to convince some scientists and science journalists that it is wise to split the difference between intellectual integrity and the fantasies of a prior age.

Because there are no easy remedies for social inequality, many scientists and public intellectuals also believe that the great masses of humanity are best kept sedated by pious delusions.
Many assert that, while they can get along just fine without an imaginary friend, most human beings will always need to believe in God.
In my experience, people holding this opinion never seem to notice how condescending, unimaginative, and pessimistic a view it is of the rest of humanity --- and of generations to come.

There are social, economic, environmental, and geopolitical costs to this strategy of benign neglect --- ranging from personal hypocrisy to public policies that needlessly undermine the health and safety of millions.
Nevertheless, many scientists seem to worry that subjecting people’s religious beliefs to criticism will start a war of ideas that science cannot win.
I believe that they are wrong.
More important, I am confident that we will eventually have no choice in the matter.
Zero-sum conflicts have a way of becoming explicit.
Here is our situation :
if the basic claims of religion are true, the scientific worldview is so blinkered and susceptible to supernatural modification as to be rendered nearly ridiculous;
if the basic claims of religion are false, most people are profoundly confused about the nature of reality, confounded by irrational hopes and fears, and tending to waste precious time and attention --- often with tragic results.
Is this really a dichotomy about which science can claim to be neutral?
The deference and condescension of most scientists on these subjects is part of a larger problem in public discourse :
people tend not to speak honestly about the nature of belief, about the invidious gulf between science and religion as modes of thought, or about the real sources of moral progress.
Whatever is true about us, ethically and spiritually, is discoverable in the present and can be talked about in terms that are not an outright affront to our growing understanding of the world.
It makes no sense at all to have the most important features of our lives anchored to divisive claims about the unique sanctity of ancient books or to rumors of ancient miracles.
There is simply no question that how we speak about human values --- and how we study or fail to study the relevant phenomena at the level of the brain --- will profoundly influence our collective future.

\newpage
\section{Chapter 1 : Moral Truths}

Many people believe that something in the last few centuries of intellectual progress prevents us from speaking in terms of ``moral truth'' and, therefore, from making cross-cultural moral judgments --- or moral judgments at all.
Having discussed this subject in a variety of public forums, I have heard from literally thousands of highly educated men and women that morality is a myth, that statements about human values are without truth conditions (and are, therefore, nonsensical), and that concepts like well-being and misery are so poorly defined, or so susceptible to personal whim and cultural influence, that it is impossible to know anything about them.
Many of these people also claim that a scientific foundation for morality would serve no purpose in any case.
They think we can combat human evil all the while knowing that our notions of ``good'' and ``evil'' are completely unwarranted.
It is always amusing when these same people then hesitate to condemn specific instances of patently abominable behavior.
I don’t think one has fully enjoyed the life of the mind until one has seen a celebrated scholar defend the ``contextual'' legitimacy of the burqa, or of female genital mutilation, a mere thirty seconds after announcing that moral relativism does nothing to diminish a person’s commitment to making the world a better place.

And so it is obvious that before we can make any progress toward a science of morality, we will have to clear some philosophical brush.
In this chapter, I attempt to do this within the limits of what I imagine to be most readers’ tolerance for such projects.
Those who leave this section with their doubts intact are encouraged to consult the endnotes.

First, I want to be very clear about my general thesis :
I am not suggesting that science can give us an evolutionary or neurobiological account of what people do in the name of ``morality.''
Nor am I merely saying that science can help us get what we want out of life.
These would be quite banal claims to make --- unless one happens to doubt the truth of evolution, the mind’s dependency on the brain, or the general utility of science.
Rather I am arguing that science can, in principle, help us understand what we \textit{should} do and \textit{should} want --- and, therefore, what \textit{other people} should do and should want in order to live the best lives possible.
My claim is that there are right and wrong answers to moral questions, just as there are right and wrong answers to questions of physics, and such answers may one day fall within reach of the maturing sciences of mind.

Once we see that a concern for well-being (defined as deeply and as inclusively as possible) is the only intelligible basis for morality and values, we will see that there must be a science of morality, whether or not we ever succeed in developing it :
because the well-being of conscious creatures depends upon how the universe is, altogether.
Given that changes in the physical universe and in our experience of it can be understood, science should increasingly enable us to answer specific moral questions.
For instance, would it be better to spend our next billion dollars eradicating racism or malaria?
Which is generally more harmful to our personal relationships, ``white'' lies or gossip?
Such questions may seem impossible to get a hold of at this moment, but they may not stay that way forever.
As we come to understand how human beings can best collaborate and thrive in this world, science can help us find a path leading away from the lowest depths of misery and toward the heights of happiness for the greatest number of people.
Of course, there will be practical impediments to evaluating the consequences of certain actions, and different paths through life may be morally equivalent (i.e., there may be many peaks on the moral landscape), but I am arguing that there are no obstacles, in principle, to our speaking about \textit{moral truth}.

It seems to me, however, that most educated, secular people (and this includes most scientists, academics, and journalists) believe that there is no such thing as moral truth --- only moral preference, moral opinion, and emotional reactions that we mistake for genuine knowledge of right and wrong.
While we can understand how human beings think and behave in the name of ``morality,'' it is widely imagined that there are no right answers to moral questions for science to discover.

Some people maintain this view by defining ``science'' in exceedingly narrow terms, as though it were synonymous with mathematical modeling or immediate access to experimental data.
However, this is to mistake science for a few of its tools.
Science simply represents our best effort to understand what is going on in this universe, and the boundary between it and the rest of rational thought cannot always be drawn.
There are many tools one must get in hand to think scientifically --- ideas about cause and effect, respect for evidence and logical coherence, a dash of curiosity and intellectual honesty, the inclination to make falsifiable predictions, etc. --- and these must be put to use long before one starts worrying about mathematical models or specific data.
Many people are also confused about what it means to speak with scientific ``objectivity'' about the human condition.
As the philosopher John Searle once pointed out, there are two very different senses of the terms ``objective'' and ``subjective.''
The first sense relates to how we know (i.e., epistemology), the second to what there is to know (i.e., ontology).
When we say that we are reasoning or speaking ``objectively,'' we generally mean that we are free of obvious bias, open to counterarguments, cognizant of the relevant facts, and so on.
This is to make a claim about how we are thinking.
In this sense, there is no impediment to our studying subjective (i.e., first-person) facts ``objectively.''

For instance, it is true to say that I am experiencing tinnitus (ringing in my ear) at this moment.
This is a subjective fact about me, but in stating this fact, I am being entirely objective :
I am not lying;
I am not exaggerating for effect;
I am not expressing a mere preference or personal bias.
I am simply stating a fact about what I am hearing at this moment.
I have also been to an otologist and had the associated hearing loss in my right ear confirmed.
No doubt, my experience of tinnitus must have an objective (third-person) cause that could be discovered (likely, damage to my cochlea).
There is simply no question that I can speak about my tinnitus in the spirit of scientific objectivity --- and, indeed, the sciences of mind are largely predicated on our being able to correlate first-person reports of subjective experience with third-person states of the brain.
This is the only way to study a phenomenon like depression :
the underlying brain states must be distinguished with reference to a person’s subjective experience.

However, many people seem to think that because moral facts relate to our experience (and are, therefore, ontologically ``subjective''), all talk of morality must be ``subjective'' in the epistemological sense (i.e., biased, merely personal, etc.).
This is simply untrue.
I hope it is clear that when I speak about ``objective'' moral truths, or about the ``objective'' causes of human well-being, I am not denying the necessarily subjective (i.e., experiential) component of the facts under discussion.
I am certainly not claiming that moral truths exist independent of the experience of conscious beings --- like the Platonic Form of the Good --- or that certain actions are intrinsically wrong.
I am simply saying that, given that there are facts --- real facts --- to be known about how conscious creatures can experience the worst possible misery and the greatest possible well-being, it is objectively true to say that there are right and wrong answers to moral questions, whether or not we can always answer these questions in practice.

And, as I have said, people consistently fail to distinguish between there being answers in practice and answers in principle to specific questions about the nature of reality.
When thinking about the application of science to questions of human well-being, it is crucial that we not lose sight of this distinction.
After all, there are countless phenomena that are subjectively real, which we can discuss objectively (i.e., honestly and rationally), but which remain impossible to describe with precision.
Consider the complete set of ``birthday wishes'' corresponding to every conscious hope that people have entertained silently while blowing out candles on birthday cakes.
Will we ever be able to retrieve these unspoken thoughts?
Of course not.
Many of us would be hard-pressed to recall even one of our own birthday wishes.
Does this mean that these wishes never existed or that we can’t make true or false statements about them?
What if I were to say that every one of these wishes was phrased in Latin, focused on improvements in solar panel technology, and produced by the activity of exactly 10,000 neurons in each person’s brain?
Is this a vacuous assertion?
No, it is quite precise and surely wrong.
But only a lunatic could believe such a thing about his fellow human beings.
Clearly, we can make true or false claims about human (and animal) subjectivity, and we can often evaluate these claims without having access to the facts in question.
This is a perfectly reasonable, scientific, and often necessary thing to do.
And yet many scientists will say that moral truths do not exist, simply because certain facts about human experience cannot be readily known, or may never be known.
As I hope to show, this misunderstanding has created tremendous confusion about the relationship between human knowledge and human values.

Another thing that makes the idea of moral truth difficult to discuss is that people often employ a double standard when thinking about consensus :
most people take scientific consensus to mean that scientific truths exist, and they consider scientific controversy to be merely a sign that further work remains to be done;
and yet many of these same people believe that moral controversy proves that there can be no such thing as moral truth, while moral consensus shows only that human beings often harbor the same biases.
Clearly, this double standard rigs the game against a universal conception of morality.

The deeper issue, however, is that truth has nothing, in principle, to do with consensus :
one person can be right, and everyone else can be wrong.
Consensus is a guide to discovering what is going on in the world, but that is all that it is.
Its presence or absence in no way constrains what may or may not be true.
There are surely physical, chemical, and biological facts about which we are ignorant or mistaken.
In speaking of ``moral truth,'' I am saying that there must be facts regarding human and animal well-being about which we can also be ignorant or mistaken.
In both cases, science --- and rational thought generally --- is the tool we can use to uncover these facts.

And here is where the real controversy begins, for many people strongly object to my claim that morality and values relate to facts about the well-being of conscious creatures.
My critics seem to think that consciousness holds no special place where values are concerned, or that any state of consciousness stands the same chance of being valued as any other.
The most common objection to my argument is some version of the following :

But you haven’t said \textit{why} the well-being of conscious beings ought to matter to us.
If someone wants to torture all conscious beings to the point of madness, what is to say that he isn’t just as ``moral'' as you are?

While I do not think anyone sincerely believes that this kind of moral skepticism makes sense, there is no shortage of people who will press this point with a ferocity that often passes for sincerity.
Let us begin with the fact of consciousness :
I think we can know, through reason alone, that consciousness is the only intelligible domain of value.
What is the alternative?
I invite you to try to think of a source of value that has absolutely nothing to do with the (actual or potential) experience of conscious beings.
Take a moment to think about what this would entail :
whatever this alternative is, it cannot affect the experience of any creature (in this life or in any other).
Put this thing in a box, and what you have in that box is --- it would seem, by \textit{definition} --- the least interesting thing in the universe.

So how much time should we spend worrying about such a transcendent source of value?
I think the time I will spend typing this sentence is already too much.
All other notions of value will bear some relationship to the actual or potential experience of conscious beings.
So my claim that consciousness is the basis of human values and morality is not an arbitrary starting point.
Now that we have consciousness on the table, my further claim is that the concept of ``well-being'' captures all that we can intelligibly value.
And ``morality'' --- whatever people’s associations with this term happen to be --- \textit{really} relates to the intentions and behaviors that affect the well-being of conscious creatures.

On this point, religious conceptions of moral law are often put forward as counterexamples :
for when asked why it is important to follow God’s law, many people will cannily say, ``for its own sake.''
Of course, it is possible to say this, but this seems neither an honest nor a coherent claim.
What if a more powerful God would punish us for eternity for following Yahweh’s law?
Would it then make sense to follow Yahweh’s law ``for its own sake''?
The inescapable fact is that religious people are as eager to find happiness and to avoid misery as anyone else :
many of them just happen to believe that the most important changes in conscious experience occur after death (i.e., in heaven or in hell).
And while Judaism is sometimes held up as an exception --- because it tends not to focus on the afterlife --- the Hebrew Bible makes it absolutely clear that Jews should follow Yahweh’s law \textit{out of concern for the negative consequences of not following it}.
People who do not believe in God or an afterlife, and yet still think it important to subscribe to a religious tradition, only believe this because living this way seems to make some positive contribution to their well-being or to the well-being of others.

Religious notions of morality, therefore, are not exceptions to our common concern for well-being.
And all other philosophical efforts to describe morality in terms of duty, fairness, justice, or some other principle that is not explicitly tied to the well-being of conscious creatures, draw upon some conception of well-being in the end.

The doubts that immediately erupt on this point invariably depend upon bizarre and restrictive notions of what the term ``well-being'' might mean.
I think there is little doubt that most of what matters to the average person --- like fairness, justice, compassion, and a general awareness of terrestrial reality --- will be integral to our creating a thriving global civilization and, therefore, to the greater well-being of humanity.
And, as I have said, there may be many different ways for individuals and communities to thrive --- many peaks on the moral landscape --- so if there is real diversity in how people can be deeply fulfilled in this life, such diversity can be accounted for and honored in the context of science.
The concept of ``well-being,'' like the concept of ``health,'' is truly open for revision and discovery.
Just how fulfilled is it possible for us to be, personally and collectively?
What are the conditions --- ranging from changes in the genome to changes in economic systems --- that will produce such happiness?
We simply do not know.

But what if certain people insist that their ``values'' or ``morality'' have nothing to do with well-being?
Or, more realistically, what if their conception of well-being is so idiosyncratic and circumscribed as to be hostile, in principle, to the well-being of all others?
For instance, what if a man like Jeffrey Dahmer says, ``The only peaks on the moral landscape that interest me are ones where I get to murder young men and have sex with their corpses.''
This possibility --- the prospect of radically different moral commitments --- is at the heart of many people’s doubts about moral truth.

Again, we should observe the double standard in place regarding the significance of consensus :
those who do not share our scientific goals have no influence on scientific discourse whatsoever;
but, for some reason, people who do not share our moral goals render us incapable of even speaking about moral truth.
It is, perhaps, worth remembering that there are trained ``scientists'' who are Biblical Creationists, and their ``scientific'' thinking is purposed toward interpreting the data of science to fit the Book of Genesis.
Such people claim to be doing ``science,'' of course, but real scientists are free, and indeed obligated, to point out that they are misusing the term.
Similarly, there are people who claim to be highly concerned about ``morality'' and ``human values,'' but when we see that their beliefs cause tremendous misery, nothing need prevent us from saying that they are misusing the term ``morality'' or that their values are distorted.
How have we convinced ourselves that, on the most important questions in human life, all views must count equally?

Consider the Catholic Church :
an organization which advertises itself as the greatest force for good and as the only true bulwark against evil in the universe.
Even among non-Catholics, its doctrines are widely associated with the concepts of ``morality'' and ``human values.''
However, the Vatican is an organization that excommunicates women for attempting to become priests but does not excommunicate male priests for raping children.
It excommunicates doctors who perform abortions to save a mother’s life --- even if the mother is a \textit{nine-year-old girl raped by her stepfather and pregnant with twins} --- but it did not excommunicate a single member of the Third Reich for committing genocide.
Are we really obliged to consider such a diabolical inversion of priorities to be evidence of an alternative ``moral'' framework?
No.

It seems clear that the Catholic Church is as misguided in speaking about the ``moral'' peril of contraception, for instance, as it would be in speaking about the ``physics'' of Transubstantiation.
In both domains, it is true to say that the Church is grotesquely confused about which things in this world are worth paying attention to.
However, many people will continue to insist that we cannot speak about moral truth, or anchor morality to a deeper concern for well-being, because concepts like ``morality'' and ``well-being'' must be defined with reference to specific goals and other criteria --- and nothing prevents people from disagreeing about these definitions.
I might claim that morality is really about maximizing well-being and that well-being entails a wide range of psychological virtues and wholesome pleasures, but someone else will be free to say that morality depends upon worshipping the gods of the Aztecs and that well-being, if it matters at all, entails always having a terrified person locked in one’s basement, waiting to be sacrificed.

Of course, goals and conceptual definitions matter.
But this holds for all phenomena and for every method we might use to study them.
My father, for instance, has been dead for twenty-five years.
What do I mean by ``dead''?
Do I mean ``dead'' with reference to specific goals?
Well, if you must, yes --- goals like respiration, energy metabolism, responsiveness to stimuli, etc.
The definition of ``life'' remains, to this day, difficult to pin down.
Does this mean we can’t study life scientifically?
No.
The science of biology thrives despite such ambiguities.
Again, the concept of ``health'' is looser still :
it, too, must be defined with reference to specific goals --- not suffering chronic pain, not always vomiting, etc. --- and these goals are continually changing.
Our notion of ``health'' may one day be defined by goals that we cannot currently entertain with a straight face (like the goal of spontaneously regenerating a lost limb).
Does this mean we can’t study health scientifically?

I wonder if there is anyone on earth who would be tempted to attack the philosophical underpinnings of medicine with questions like: ``What about all the people who don’t share your goal of avoiding disease and early death?
Who is to say that living a long life free of pain and debilitating illness is `healthy'?
What makes you think that you could convince a person suffering from fatal gangrene that he is not as healthy as you are?''
And yet these are precisely the kinds of objections I face when I speak about morality in terms of human and animal well-being.
Is it possible to voice such doubts in human speech?
Yes.
But that doesn’t mean we should take them seriously.
One of my critics put the concern this way :
``Morals are relative to the time and place in which they appear.
If you do not already accept well-being as a value, then there seems to be no argument for why one \textit{should} promote well-being.''
As proof of this assertion, he observed that I would be unable to convince the Taliban that they value the wrong things.
By this standard, however, the truths of science are also ``relative to the time and place in which they appear,'' and there is no way to convince someone who does not value empirical evidence that he should value it.
Despite 150 years of working at it, we still can’t convince a majority of Americans that evolution is a fact.
Does this mean biology isn’t a proper science?

Everyone has an intuitive ``physics,'' but much of our intuitive physics is wrong (with respect to the goal of describing the behavior of matter).
Only physicists have a deep understanding of the laws that govern the behavior of matter in our universe.
I am arguing that everyone also has an intuitive ``morality,'' but much of our intuitive morality is clearly wrong (with respect to the goal of maximizing personal and collective well-being).
And only genuine moral experts would have a deep understanding of the causes and conditions of human and animal well-being.
Yes, we must have a goal to define what counts as ``right'' or ``wrong'' when speaking about physics or morality, but this criterion visits us equally in both domains.
And yes, I think it is quite clear that members of the Taliban are seeking well-being in this world (as well as hoping for it in the next).
But their religious beliefs have led them to create a culture that is almost perfectly hostile to human flourishing.
Whatever they \textit{think} they want out of life --- like keeping all women and girls subjugated and illiterate --- they simply do not understand how much better life would be for them if they had different priorities.

Science cannot tell us why, \textit{scientifically}, we should value health.
But once we admit that health is the proper concern of medicine, we can then study and promote it through science.
Medicine can resolve specific questions about human health --- and it can do this even while the very definition of ``health'' continues to change.
Indeed, the science of medicine can make marvelous progress without knowing how much its own progress will alter our conception of health in the future.

I think our concern for well-being is even less in need of justification than our concern for health is --- as health is merely one of its many facets.
And once we begin thinking seriously about human well-being, we will find that science can resolve specific questions about morality and human values, even while our conception of ``well-being'' evolves.

It is essential to see that the demand for \textit{radical} justification leveled by the moral skeptic could not be met by any branch of science.
Science is defined with reference to the goal of understanding the processes at work in the universe.
Can we justify this goal scientifically?
Of course not.
Does this make science itself unscientific?
If so, we appear to have pulled ourselves \textit{down} by our bootstraps.

It would be impossible to prove that our definition of science is correct, because our standards of proof will be built into any proof we would offer.
What evidence could prove that we should value evidence?
What logic could demonstrate the importance of logic?
We might observe that standard science is better at predicting the behavior of matter than Creationist ``science'' is.
But what could we say to a ``scientist'' whose only goal is to authenticate the Word of God?
Here, we seem to reach an impasse.
And yet, no one thinks that the failure of standard science to silence all possible dissent has any significance whatsoever;
why should we demand more of a science of morality?
Many moral skeptics piously cite Hume’s is/ought distinction as though it were well known to be the last word on the subject of morality until the end of the world.
They insist that notions of what we ought to do or value can be justified only in terms of other ``oughts,'' never in terms of facts about the way the world is.
After all, in a world of physics and chemistry, how could things like moral obligations or values really exist?
How could it be objectively true, for instance, that we ought to be kind to children?

But this notion of ``ought'' is an artificial and needlessly confusing way to think about moral choice.
In fact, it seems to be another dismal product of Abrahamic religion --- which, strangely enough, now constrains the thinking of even atheists.
If this notion of ``ought'' means anything we can possibly care about, it must translate into a concern about the actual or potential experience of conscious beings (either in this life or in some other).
For instance, to say that we ought to treat children with kindness seems identical to saying that everyone will tend to be better off if we do.
The person who claims that he does not want to be better off is either wrong about what he does, in fact, want (i.e., he doesn’t know what he’s missing), or he is lying, or he is not making sense.
The person who insists that he is committed to treating children with kindness for reasons that have nothing to do with anyone’s well-being is also not making sense.
It is worth noting in this context that the God of Abraham never told us to treat children with kindness, but He did tell us to kill them for talking back to us (Exodus 21:15, Leviticus 20:9, Deuteronomy 21:18–21, Mark 7:9–13, and Matthew 15:4–7).
And yet everyone finds this ``moral'' imperative perfectly insane.
Which is to say that no one --- not even fundamentalist Christians and orthodox Jews --- can so fully ignore the link between morality and human well-being as to be truly bound by God’s law.

\subsection{The Worst Possible Misery for Everyone}

I have argued that values only exist relative to actual and potential changes in the well-being of conscious creatures.
However, as I have said, many people seem to have strange associations with the concept of ``well-being'' --- imagining that it must be at odds with principles like justice, autonomy, fairness, scientific curiosity, etc., when it simply isn’t.
They also worry that the concept of “well-being” is poorly defined.
Again, I have indicated why I do not think this is a problem (just as it’s not a problem with concepts like “life” and “health”).
However, it is also useful to notice that a universal morality can be defined with reference to the negative end of the spectrum of conscious experience : I
refer to this extreme as ``the worst possible misery for everyone.''

Even if each conscious being has a unique nadir on the moral landscape, we can still conceive of a state of the universe in which everyone suffers as much as he or she (or it) possibly can.
If you think we cannot say this would be ``bad,'' then I don’t know what you could mean by the word ``bad'' (and I don’t think you know what you mean by it either).
Once we conceive of ``the worst possible misery for everyone,'' then we can talk about taking incremental steps toward this abyss :
What could it mean for life on earth to get worse for all human beings simultaneously?
Notice that this need have nothing to do with people enforcing their culturally conditioned moral precepts.
Perhaps a neurotoxic dust could fall to earth from space and make everyone extremely uncomfortable.
All we need imagine is a scenario in which everyone loses a little, or a lot, without there being compensatory gains (i.e., no one learns any important lessons, no one profits from others’ losses, etc.).
It seems uncontroversial to say that a change that leaves everyone worse off, by any rational standard, can be reasonably called ``bad,'' if this word is to have any meaning at all.

We simply must stand somewhere. I am arguing that, in the moral sphere, it is safe to begin with the premise that it is good to avoid behaving in such a way as to produce the worst possible misery for everyone.
I am not claiming that most of us personally care about the experience of all conscious beings;
I am saying that a universe in which all conscious beings suffer the worst possible misery is worse than a universe in which they experience well-being.
This is all we need to speak about ``moral truth'' in the context of science.
Once we admit that the extremes of absolute misery and absolute flourishing --- whatever these states amount to for each particular being in the end --- are different and dependent on facts about the universe, then we have admitted that there are right and wrong answers to questions of morality.

Granted, genuine ethical difficulties arise when we ask questions like, ``How much should I care about other people’s children?
How much should I be willing to sacrifice, or demand that my own children sacrifice, in order to help other people in need?''
We are not, by nature, impartial --- and much of our moral reasoning must be applied to situations in which there is tension between our concern for ourselves, or for those closest to us, and our sense that it would be better to be more committed to helping others.
And yet ``better'' must still refer, in this context, to positive changes in the experience of sentient creatures.

Imagine if there were only two people living on earth :
we can call them Adam and Eve.
Clearly, we can ask how these two people might maximize their well-being.
Are there wrong answers to this question?
Of course.
(Wrong answer number 1: smash each other in the face with a large rock.)
And while there are ways for their personal interests to be in conflict, most solutions to the problem of how two people can thrive on earth will not be zero-sum.
Surely the best solutions will not be zero-sum.
Yes, both of these people could be blind to the deeper possibilities of collaboration :
each might attempt to kill and eat the other, for instance.
Would they be wrong to behave this way?
Yes, if by ``wrong'' we mean that they would be forsaking far deeper and more durable sources of satisfaction.
It seems uncontroversial to say that a man and woman alone on earth would be better off if they recognized their common interests --- like getting food, building shelter, and defending themselves against larger predators.
If Adam and Eve were industrious enough, they might realize the benefits of exploring the world, begetting future generations of humanity, and creating technology, art, and medicine.
Are there good and bad paths to take across this landscape of possibilities?
Of course. In fact, there are, by definition, paths that lead to the worst misery and paths that lead to the greatest fulfillment possible for these two people --- given the structure of their respective brains, the immediate facts of their environment, and the laws of Nature.
The underlying facts here are the facts of physics, chemistry, and biology as they bear on the experience of the only two people in existence.
Unless the human mind is fully separable from the principles of physics, chemistry, and biology, any fact about Adam and Eve’s subjective experience (morally salient or not) is a fact about (part of) the universe.

In talking about the causes of Adam and Eve’s first-person experience, we are talking about an extraordinarily complex interplay between brain states and environmental stimuli.
However complex these processes are, it is clearly possible to understand them to a greater or lesser degree (i.e., there are right and wrong answers to questions about Adam’s and Eve’s well-being).
Even if there are a thousand different ways for these two people to thrive, there will be many ways for them not to thrive --- and the differences between luxuriating on a peak of well-being and languishing in a valley of internecine horror will translate into facts that can be scientifically understood.
Why would the difference between right and wrong answers suddenly disappear once we add 6.7 billion more people to this experiment?

Grounding our values in a continuum of conscious states --- one that has the worst possible misery for everyone at its depths and differing degrees of well-being at all other points --- seems like the only legitimate context in which to conceive of values and moral norms.
Of course, anyone who has an alternative set of moral axioms is free to put them forward, just as they are free to define ``science'' any way they want.
But some definitions will be useless, or worse --- and many current definitions of ``morality'' are so bad that we can know, far in advance of any breakthrough in the sciences of mind, that they have no place in a serious conversation about how we should live in this world.
The Knights of the Ku Klux Klan have nothing meaningful to say about particle physics, cell physiology, epidemiology, linguistics, economic policy, etc.
How is their ignorance any less obvious on the subject of human well-being?

The moment we admit that consciousness is the context in which any discussion of values makes sense, we must admit that there are facts to be known about how the experience of conscious creatures can change.
Human and animal well-being are natural phenomena.
As such, they can be studied, in principle, with the tools of science and spoken about with greater or lesser precision.
Do pigs suffer more than cows do when being led to slaughter?
Would humanity suffer more or less, on balance, if the United States unilaterally gave up all its nuclear weapons?
Questions like these are very difficult to answer.
But this does not mean that they don’t have answers.

The fact that it could be difficult or impossible to know exactly how to maximize human well-being does not mean that there are no right or wrong ways to do this --- nor does it mean that we cannot exclude certain answers as obviously bad.
For instance, there is often a tension between the autonomy of the individual and the common good, and many moral problems turn on just how to prioritize these competing values.
However, autonomy brings obvious benefit to people and is, therefore, an important component of the common good.
The fact that it might be difficult to decide exactly how to balance individual rights against collective interests, or that there might be a thousand equivalent ways of doing this, does not mean that there aren’t objectively terrible ways of doing this.
The difficulty of getting precise answers to certain moral questions does not mean that we must hesitate to condemn the morality of the Taliban --- not just personally, but from the point of view of science.
The moment we admit that we know anything about human well-being scientifically, we must admit that certain individuals or cultures can be absolutely wrong about it.

\subsection{Moral Blindness in the Name of ``Tolerance''}

There are very practical concerns that follow from the glib idea that anyone is free to value anything --- the most consequential being that it is precisely what allows highly educated, secular, and otherwise well-intentioned people to pause thoughtfully, and often interminably, before condemning practices like compulsory veiling, genital excision, bride burning, forced marriage, and other cheerful products of alternative ``morality'' found elsewhere in the world.
Fanciers of Hume's is/ought distinction never seem to realize what the stakes are, and they do not see how abject failures of compassion are enabled by this intellectual ``tolerance'' of moral difference.
While much of the debate on these issues must be had in academic terms, this is not merely an academic debate.
There are girls getting their faces burned off with acid at this moment for daring to learn to read, or for not consenting to marry men they have never met, or even for the ``crime'' of getting raped.
The amazing thing is that some Western intellectuals won’t even blink when asked to defend these practices on philosophical grounds.
I once spoke at an academic conference on themes similar to those discussed here.
Near the end of my lecture, I made what I thought would be a quite incontestable assertion :
We already have good reason to believe that certain cultures are less suited to maximizing well-being than others.
I cited the ruthless misogyny and religious bamboozlement of the Taliban as an example of a worldview that seems less than perfectly conducive to human flourishing.

As it turns out, to denigrate the Taliban at a scientific meeting is to court controversy.
At the conclusion of my talk, I fell into debate with another invited speaker, who seemed, at first glance, to be very well positioned to reason effectively about the implications of science for our understanding of morality.
In fact, this person has since been appointed to the President’s Commission for the Study of Bioethical Issues and is now one of only thirteen people who will advise President Obama on ``issues that may emerge from advances in biomedicine and related areas of science and technology'' in order to ensure that ``scientific research, health care delivery, and technological innovation are conducted in an ethically responsible manner.''
Here is a snippet of our conversation, more or less verbatim :

She:
What makes you think that science will ever be able to say that forcing women to wear burqas is wrong?

Me:
Because I think that right and wrong are a matter of increasing or decreasing well-being --- and it is obvious that forcing half the population to live in cloth bags, and beating or killing them if they refuse, is not a good strategy for maximizing human well-being.

She:
But that’s only your opinion.

Me:
Okay \dots Let’s make it even simpler.
What if we found a culture that ritually blinded every third child by literally plucking out his or her eyes at birth, would you then agree that we had found a culture that was needlessly diminishing human well-being?

She:
It would depend on why they were doing it.

Me [slowly returning my eyebrows from the back of my head]:
Let’s say they were doing it on the basis of religious superstition.
In their scripture, God says, ``Every third must walk in darkness.''

She:
Then you could never say that they were wrong.

Such opinions are not uncommon in the Ivory Tower.
I was talking to a woman (it’s hard not to feel that her gender makes her views all the more disconcerting) who had just delivered an entirely lucid lecture on some of the moral implications of recent advances in neuroscience.
She was concerned that our intelligence services might one day use neuroimaging technology for the purposes of lie detection, which she considered a likely violation of cognitive liberty.
She was especially exercised over rumors that our government might have exposed captured terrorists to aerosols containing the hormone oxytocin in an effort to make them more cooperative.
Though she did not say it, I suspect that she would even have opposed subjecting these prisoners to the smell of freshly baked bread, which has been shown to have a similar effect.
While listening to her talk, as yet unaware of her liberal views on compulsory veiling and ritual enucleation, I thought her slightly overcautious, but a basically sane and eloquent authority on scientific ethics.
I confess that once we did speak, and I peered into the terrible gulf that separated us on these issues, I found that I could not utter another word to her.
In fact, our conversation ended with my blindly enacting two neurological cliches :
my jaw quite literally dropped open, and I spun on my heels before walking away.

While human beings have different moral codes, each competing view presumes its own universality.
This seems to be true even of moral relativism.
While few philosophers have ever answered to the name of ``moral relativist,'' it is by no means uncommon to find local eruptions of this view whenever scientists and other academics encounter moral diversity.
Forcing women and girls to wear burqas may be wrong in Boston or Palo Alto, so the argument will run, but we cannot say that it is wrong for Muslims in Kabul.
To demand that the proud denizens of an ancient culture conform to our view of gender equality would be culturally imperialistic and philosophically naive.
This is a surprisingly common view, especially among anthropologists.

Moral relativism, however, tends to be self-contradictory.
Relativists may say that moral truths exist only relative to a specific cultural framework --- but this claim about the status of moral truth purports to be true across all possible frameworks.
In practice, relativism almost always amounts to the claim that we should be tolerant of moral difference because no moral truth can supersede any other.
And yet this commitment to tolerance is not put forward as simply one relative preference among others deemed equally valid.
Rather, tolerance is held to be more in line with the (universal) truth about morality than intolerance is.
The contradiction here is unsurprising.
Given how deeply disposed we are to make universal moral claims, I think one can reasonably doubt whether any consistent moral relativist has ever existed.

Moral relativism is clearly an attempt to pay intellectual reparations for the crimes of Western colonialism, ethnocentrism, and racism.
This is, I think, the only charitable thing to be said about it.
I hope it is clear that I am not defending the idiosyncrasies of the West as any more enlightened, in principle, than those of any other culture.
Rather, I am arguing that the most basic facts about human flourishing must transcend culture, just as most other facts do.
And if there are facts that are truly a matter of cultural construction --- if, for instance, learning a specific language or tattooing your face fundamentally alters the possibilities of human experience --- well, then these facts also arise from (neurophysiological) processes that transcend culture.

In his wonderful book The \textit{Blank Slate}, Steven Pinker includes a quotation from the anthropologist Donald Symons that captures the problem of multiculturalism especially well :

If only one person in the world held down a terrified, struggling, screaming little girl, cut off her genitals with a septic blade, and sewed her back up, leaving only a tiny hole for urine and menstrual flow, the only question would be how severely that person should be punished, and whether the death penalty would be a sufficiently severe sanction.
But when millions of people do this, instead of the enormity being magnified millions-fold, suddenly it becomes ``culture,'' and thereby magically becomes less, rather than more, horrible, and is even defended by some Western ``moral thinkers,'' including feminists.

It is precisely such instances of learned confusion (one is tempted to say ``learned psychopathy'') that lend credence to the claim that a universal morality requires the support of faith-based religion.
The categorical distinction between facts and values has opened a sinkhole beneath secular liberalism --- leading to moral relativism and masochistic depths of political correctness.
Think of the champions of ``tolerance'' who reflexively blamed Salman Rushdie for his fatwa, or Ayaan Hirsi Ali for her ongoing security concerns, or the Danish cartoonists for their ``controversy,'' and you will understand what happens when educated liberals think there is no universal foundation for human values.
Among conservatives in the West, the same skepticism about the power of reason leads, more often than not, directly to the feet of Jesus Christ, Savior of the Universe.
The purpose of this book is to help cut a third path through this wilderness.

\subsection{Moral Science}

Charges of ``scientism'' cannot be long in coming.
No doubt, there are still some people who will reject any description of human nature that was not first communicated in iambic pentameter.
Many readers may also fear that the case I am making is vaguely, or even explicitly, utopian.
It isn’t, as should become clear in due course.

However, other doubts about the authority of science are even more fundamental.
There are academics who have built entire careers on the allegation that the foundations of science are rotten with bias --- sexist, racist, imperialist, Northern, etc.
Sandra Harding, a feminist philosopher of science, is probably the most famous proponent of this view.
On her account, these prejudices have driven science into an epistemological cul-de-sac called ``weak objectivity.''
To remedy this dire situation, Harding recommends that scientists immediately give ``feminist'' and ``multicultural'' epistemologies their due.

First, let’s be careful not to confuse this quite crazy claim for its sane cousin :
There is no question that scientists have occasionally demonstrated sexist and racist biases.
The composition of some branches of science is still disproportionately white and male (though some are now disproportionately female), and one can reasonably wonder whether bias is the cause.
There are also legitimate questions to be asked about the direction and application of science :
in medicine, for instance, it seems clear that women’s health issues have been sometimes neglected because the prototypical human being has been considered male.
One can also argue that the contributions of women and minority groups to science have occasionally been ignored or undervalued :
the case of Rosalind Franklin standing in the shadows of Crick and Watson might be an example of this.
But none of these facts, alone or in combination, or however multiplied, remotely suggests that our notions of scientific objectivity are vitiated by racism or sexism.

Is there really such a thing as a feminist or multicultural epistemology?
Harding’s case is not helped when she finally divulges that there is not just one feminist epistemology, but many.
On this view, why was Hitler’s notion of ``Jewish physics'' (or Stalin’s idea of ``capitalist biology'') anything less than a thrilling insight into the richness of epistemology?
Should we now consider the possibility of not only Jewish physics, but of Jewish women’s physics?
How could such a balkanization of science be a step toward ``strong objectivity''?
And if political inclusiveness is our primary concern, where could such efforts to broaden our conception of scientific truth possibly end?
Physicists tend to have an unusual aptitude for complex mathematics, and anyone who doesn’t cannot expect to make much of a contribution to the field.
Why not remedy this situation as well?
Why not create an epistemology for physicists who failed calculus?
Why not be bolder still and establish a branch of physics for people suffering from debilitating brain injuries?
Who could reasonably expect that such efforts at inclusiveness would increase our understanding of a phenomenon like gravity?
As Steven Weinberg once said regarding similar doubts about the objectivity of science, ``You have to be very learned to be that wrong.''
Indeed, one does --- and many are.

There is no denying, however, that the effort to reduce all human values to biology can produce howlers.
For instance, when the entomologist E. O. Wilson (in collaboration with the philosopher Michael Ruse) wrote that ``morality, or more strictly our belief in morality, is merely an adaptation put in place to further our reproductive ends,'' the philosopher Daniel Dennett rightly dismissed it as ``nonsense.''
The fact that our moral intuitions probably conferred some adaptive advantage upon our ancestors does not mean that the present purpose of morality is successful reproduction, or that ``our belief in morality'' is just a useful delusion.
(Is the purpose of astronomy successful reproduction?
What about the practice of contraception?
Is that all about reproduction, too?)
Nor does it mean that our notion of ``morality'' cannot grow deeper and more refined as our understanding of ourselves develops.

Many universal features of human life need not have been selected for at all;
they may simply be, as Dennett says, ``good tricks'' communicated by culture or ``forced moves'' that naturally emerge out of the regularities in our world.
As Dennett says, it is doubtful that there is a gene for knowing that you should throw a spear ``pointy end first.''
And it is, likewise, doubtful that our ancestors had to spend much time imparting this knowledge to each successive generation.
We have good reason to believe that much of what we do in the name of ``morality'' --- decrying sexual infidelity, punishing cheaters, valuing cooperation, etc. --- is borne of unconscious processes that were shaped by natural selection.

But this does not mean that evolution designed us to lead deeply fulfilling lives.
Again, in talking about a science of morality, I am not referring to an evolutionary account of all the cognitive and emotional processes that govern what people do when they say they are being ``moral'';
I am referring to the totality of scientific facts that govern the range of conscious experiences that are possible for us.
To say that there are truths about morality and human values is simply to say that there are facts about well-being that await our discovery --- regardless of our evolutionary history.
While such facts necessarily relate to the experience of conscious beings, they cannot be the mere invention of any person or culture.

It seems to me, therefore, that there are at least three projects that we should not confuse:

\begin{enumerate}
    \item We can explain why people tend to follow certain patterns of thought and behavior (many of them demonstrably silly and harmful) in the name of ``morality.''
    \item We can think more clearly about the nature of moral truth and determine which patterns of thought and behavior we should follow in the name of ``morality.''
    \item We can convince people who are committed to silly and harmful patterns of thought and behavior in the name of ``morality'' to break these commitments and to live better lives.
\end{enumerate}

These are distinct and independently worthy endeavors.
Most scientists who study morality in evolutionary, psychological, or neurobiological terms are exclusively devoted to the first project :
their goal is to describe and understand how people think and behave in light of morally salient emotions like anger, disgust, empathy, love, guilt, humiliation, etc.
This research is fascinating, of course, but it is not my focus.
And while our common evolutionary origins and resultant physiological similarity to one another suggest that human well-being will admit of general principles that can be scientifically understood, I consider this first project all but irrelevant to projects 2 and 3.
In the past, I have found myself in conflict with some of the leaders in this field because many of them, like the psychologist Jonathan Haidt, believe that this first project represents the only legitimate point of contact between science and morality.

I happen to believe that the third project --- changing people’s ethical commitments --- is the most important task facing humanity in the twenty-first century.
Nearly every other important goal --- from combating climate change, to fighting terrorism, to curing cancer, to saving the whales --- falls within its purview.
Of course, moral persuasion is a difficult business, but it strikes me as especially difficult if we haven’t figured out in what sense moral truths exist.
Hence, my main focus is on project 2.
To see the difference between these three projects, it is best to consider specific examples :
we can, for instance, give a plausible evolutionary account of why human societies have tended to treat women as the property of men (1);
it is, however, quite another thing to give a scientific account of whether, why, and to what degree human societies change for the better when they outgrow this tendency (2);
it is yet another thing altogether to decide how best to change people’s attitudes at this moment in history and to empower women on a global scale (3).
It is easy to see why the study of the evolutionary origins of ``morality'' might lead to the conclusion that morality has nothing at all to do with Truth.
If morality is simply an adaptive means of organizing human social behavior and mitigating conflict, there would be no reason to think that our current sense of right and wrong would reflect any deeper understanding about the nature of reality.
Hence, a narrow focus explaining why people think and behave as they do can lead a person to find the idea of ``moral truth'' literally unintelligible.
But notice that the first two projects give quite different accounts of how ``morality'' fits into the natural world.
In 1, ``morality'' is the collection of impulses and behaviors (along with their cultural expressions and neurobiological underpinnings) that have been hammered into us by evolution.
In 2, ``morality'' refers to the impulses and behaviors we can follow so as to maximize our well-being in the future.
To give a concrete example :
Imagine that a handsome stranger tries to seduce another man’s wife at the gym.
When the woman politely informs her admirer that she is married, the cad persists, as though a happy marriage could be no impediment to his charms.
The woman breaks off the conversation soon thereafter, but far less abruptly than might have been compatible with the laws of physics.

I write now, in the rude glare of recent experience.
I can say that when my wife reported these events to me yesterday, they immediately struck me as morally salient.
In fact, she had not completed her third sentence before the dark fluids of moral indignation began coursing through my brain --- jealousy, embarrassment, anger, etc. --- albeit only at a trickle.
First, I was annoyed by the man’s behavior --- and had I been present to witness it, I suspect that my annoyance would have been far greater.
If this Don Juan had been as dismissive of me in my presence as he was in my absence, I could imagine how such an encounter could result in physical violence.
No evolutionary psychologist would find it difficult to account for my response to this situation --- and almost all scientists who study ``morality'' would confine their attention to this set of facts :
my inner ape had swung into view, and any thoughts I might entertain about ``moral truth'' would be linguistic effluvium masking far more zoological concerns.
I am the product of an evolutionary history in which every male of the species has had to guard against squandering his resources on another man’s offspring.
Had we scanned my brain and correlated my subjective feelings with changes in my neurophysiology, the scientific description of these events would be nearly complete.
So ends project 1.
But there are many different ways for an ape to respond to the fact that other apes find his wife desirable.
Had this happened in a traditional honor culture, the jealous husband might beat his wife, drag her to the gym, and force her to identify her suitor so that he could put a bullet in his brain.
In fact, in an honor society, the employees of the gym might sympathize with this project and help to organize a proper duel.
Or perhaps the husband would be satisfied to act more obliquely, killing one of his rival’s relatives and initiating a classic blood feud.
In either case, assuming he didn’t get himself killed in the process, he might then murder his wife for emphasis, leaving his children motherless.
There are many communities on earth where men commonly behave this way, and hundreds of millions of boys are beginning to run this ancient software on their brains even now.

However, my own mind shows some precarious traces of civilization :
one being that I view the emotion of jealously with suspicion.
What is more, I happen to love my wife and genuinely want her to be happy, and this entails a certain empathetic understanding of her point of view.
Given a moment to think about it, I can feel glad that her self-esteem received a boost from this man’s attention;
I can also feel compassion for the fact that, after recently having our first child, her self-esteem needed any boost at all.
I also know that she would not want to be rude, and that this probably made her somewhat slow to extricate herself from a conversation that had taken a wrong turn.
And I am under no illusions that I am the only man on earth whom she will find attractive, or momentarily distracting, nor do I imagine that her devotion to me should consist in this impossible narrowing of her focus.
And how do I feel about the man?
Well, I still find his behavior objectionable --- because I cannot sympathize with his effort to break up a marriage, and I know that I would not behave as he did --- but I sympathize with everything else he must have felt, because I also happen to think that my wife is beautiful, and I know what it’s like to be a single ape in the jungle.
Most important, however, I value my own well-being, as well as that of my wife and daughter, and I want to live in a society that maximizes the possibility of human well-being generally.
Here begins project 2 :
Are there right and wrong answers to the question of how to maximize well-being?
How would my life have been affected if I had killed my wife in response to this episode?
We do not need a completed neuroscience to know that my happiness, as well as that of many other people, would have been profoundly diminished.
And what about the collective well-being of people in an honor society that might support such behavior?
It seems to me that members of these societies are obviously worse off.
If I am wrong about this, however, and there are ways to organize an honor culture that allow for precisely the same level of human flourishing enjoyed elsewhere --- then so be it.
This would represent another peak on the moral landscape.
Again, the existence of multiple peaks would not render the truths of morality merely subjective.

The framework of a moral landscape guarantees that many people will have flawed conceptions of morality, just as many people have flawed conceptions of physics.
Some people think ``physics'' includes (or validates) practices like astrology, voodoo, and homeopathy.
These people are, by all appearances, simply wrong about physics.
In the United States, a majority of people (57 percent) believe that preventing homosexuals from marrying is a ``moral'' imperative.
However, if this belief rests on a flawed sense of how we can maximize our well-being, such people may simply be wrong about morality.
And the fact that millions of people use the term ``morality'' as a synonym for religious dogmatism, racism, sexism, or other failures of insight and compassion should not oblige us to merely accept their terminology until the end of time.

What will it mean for us to acquire a deep, consistent, and fully scientific understanding of the human mind?
While many of the details remain unclear, the challenge is for us to begin speaking sensibly about right and wrong, and good and evil, given what we already know about our world.
Such a conversation seems bound to shape our morality and public policy in the years to come.

\newpage
\section{Chapter 2 : Good And Evil}

There may be nothing more important than human cooperation.
Whenever more pressing concerns seem to arise --- like the threat of a deadly pandemic, an asteroid impact, or some other global catastrophe --- human cooperation is the only remedy (if a remedy exists).
Cooperation is the stuff of which meaningful human lives and viable societies are made.
Consequently, few topics will be more relevant to a maturing science of human well-being.

Open a newspaper, today or any day for the rest of your life, and you will witness failures of human cooperation, great and small, announced from every corner of the world.
The results of these failures are no less tragic for being utterly commonplace :
deception, theft, violence, and their associated miseries arise in a continuous flux of misspent human energy.
When one considers the proportion of our limited time and resources that must be squandered merely to guard against theft and violence (to say nothing of addressing their effects), the problem of human cooperation seems almost the only problem worth thinking about.
``Ethics'' and ``morality'' (I use these terms interchangeably) are the names we give to our deliberate thinking on these matters.
Clearly, few subjects have greater bearing upon the question of human well-being.

As we better understand the brain, we will increasingly understand all of the forces --- kindness, reciprocity, trust, openness to argument, respect for evidence, intuitions of fairness, impulse control, the mitigation of aggression, etc. --- that allow friends and strangers to collaborate successfully on the common projects of civilization.
Understanding ourselves in this way, and using this knowledge to improve human life, will be among the most important challenges to science in the decades to come.

Many people imagine that the theory of evolution entails selfishness as a biological imperative.
This popular misconception has been very harmful to the reputation of science.
In truth, human cooperation and its attendant moral emotions are fully compatible with biological evolution.
Selection pressure at the level of ``selfish'' genes would surely incline creatures like ourselves to make sacrifices for our relatives, for the simple reason that one’s relatives can be counted on to share one’s genes :
while this truth might not be obvious through introspection, your brother’s or sister’s reproductive success is, in part, your own.
This phenomenon, known as kin selection, was not given a formal analysis until the 1960s in the work of William Hamilton, but it was at least implicit in the understanding of earlier biologists.
Legend has it that J. B. S. Haldane was once asked if he would risk his life to save a drowning brother, to which he quipped, ``No, but I would save two brothers or eight cousins.''

The work of evolutionary biologist Robert Trivers on reciprocal altruism has gone a long way toward explaining cooperation among unrelated friends and strangers.
Trivers’s model incorporates many of the psychological and social factors related to altruism and reciprocity, including :
friendship, moralistic aggression (i.e., the punishment of cheaters), guilt, sympathy, and gratitude, along with a tendency to deceive others by mimicking these states.
As first suggested by Darwin, and recently elaborated by the psychologist Geoffrey Miller, sexual selection may have further encouraged the development of moral behavior.
Because moral virtue is attractive to both sexes, it might function as a kind of peacock’s tail:
costly to produce and maintain, but beneficial to one’s genes in the end.

Clearly, our selfish and selfless interests do not always conflict.
In fact, the well-being of others, especially those closest to us, is one of our primary (and, indeed, most selfish) interests.
While much remains to be understood about the biology of our moral impulses, kin selection, reciprocal altruism, and sexual selection explain how we have evolved to be, not merely atomized selves in thrall to our self-interest, but social selves disposed to serve a common interest with others.

Certain biological traits appear to have been shaped by, and to have further enhanced, the human capacity for cooperation.
For instance, unlike the rest of the earth’s creatures, including our fellow primates, the sclera of our eyes (the region surrounding the colored iris) is white and exposed.
This makes the direction of the human gaze very easy to detect, allowing us to notice even the subtlest shifts in one another’s visual attention.
The psychologist Michael Tomasello suggests the following adaptive logic :

If I am, in effect, advertising the direction of my eyes, I must be in a social environment full of others who are not often inclined to take advantage of this to my detriment --- by, say, beating me to the food or escaping aggression before me.
Indeed, I must be in a cooperative social environment in which others following the direction of my eyes somehow benefits me.

Tomasello has found that even twelve-month old children will follow a person’s gaze, while chimpanzees tend to be interested only in head movements.
He suggests that our unique sensitivity to gaze direction facilitated human cooperation and language development.

While each of us is selfish, we are not merely so.
Our own happiness requires that we extend the circle of our self-interest to others --- to family, friends, and even to perfect strangers whose pleasures and pains matter to us.
While few thinkers have placed greater focus on the role that competing self-interests play in society, even Adam Smith recognized that each of us cares deeply about the happiness of others.
He also recognized, however, that our ability to care about others has its limits and that these limits are themselves the object of our personal and collective concern :

Let us suppose that the great empire of China, with all its myriads of inhabitants, was suddenly swallowed up by an earthquake, and let us consider how a man of humanity in Europe, who had no sort of connection with that part of the world, would be affected upon receiving intelligence of this dreadful calamity.
He would, I imagine, first of all, express very strongly his sorrow for the misfortune of that unhappy people, he would make many melancholy reflections upon the precariousness of human life, and the vanity of all the labours of man, which could thus be annihilated in a moment.
He would too, perhaps, if he was a man of speculation, enter into many reasonings concerning the effects which this disaster might produce upon the commerce of Europe, and the trade and business of the world in general.
And when all this fine philosophy was over, when all these humane sentiments had been once fairly expressed, he would pursue his business or his pleasure, take his repose or his diversion, with the same ease and tranquility, as if no such accident had happened.
The most frivolous disaster which could befall himself would occasion a more real disturbance.
If he was to lose his little finger tomorrow, he would not sleep tonight;
but, provided he never saw them, he will snore with the most profound security over the ruin of a hundred millions of his brethren, and the destruction of that immense multitude seems plainly an object less interesting to him, than this paltry misfortune of his own.
To prevent, therefore, this paltry misfortune to himself, would a man of humanity be willing to sacrifice the lives of a hundred millions of his brethren, provided he had never seen them?
Human nature startles with horror at the thought, and the world, in its greatest depravity and corruption, never produced such a villain as could be capable of entertaining it.
But what makes this difference?

Smith captures the tension between our reflexive selfishness and our broader moral intuitions about as well as anyone can here.
The truth about us is plain to see :
most of us are powerfully absorbed by selfish desires almost every moment of our lives;
our attention to our own pains and pleasures could scarcely be more acute;
only the most piercing cries of anonymous suffering capture our interest, and then fleetingly.
And yet, when we consciously reflect on what we should do, an angel of beneficence and impartiality seems to spread its wings within us :
we genuinely want fair and just societies;
we want others to have their hopes realized;
we want to leave the world better than we found it.

Questions of human well-being run deeper than any explicit code of morality.
Morality --- in terms of consciously held precepts, social contracts, notions of justice, etc. --- is a relatively recent development.
Such conventions require, at a minimum, complex language and a willingness to cooperate with strangers, and this takes us a stride or two beyond the Hobbesian ``state of nature.''
However, any biological changes that served to mitigate the internecine misery of our ancestors would fall within the scope of an analysis of morality as a guide to personal and collective well-being.
To simplify matters enormously :

\begin{enumerate}
    \item Genetic changes in the brain gave rise to social emotions, moral intuitions, and language \dots
    \item These allowed for increasingly complex cooperative behavior, the keeping of promises, concern about one’s reputation, etc \dots
    \item Which became the basis for cultural norms, laws, and social institutions whose purpose has been to render this growing system of cooperation durable in the face of countervailing forces.
\end{enumerate}

Some version of this progression has occurred in our case, and each step represents an undeniable enhancement of our personal and collective well-being.
To be sure, catastrophic regressions are always possible.
We could, either by design or negligence, employ the hard-won fruits of civilization, and the emotional and social leverage wrought of millennia of biological and cultural evolution, to immiserate ourselves more fully than unaided Nature ever could.
Imagine a global North Korea, where the better part of a starving humanity serve as slaves to a lunatic with bouffant hair :
this might be worse than a world filled merely with warring australopithecines.
What would ``worse'' mean in this context?
Just what our intuitions suggest :
more painful, less satisfying, more conducive to terror and despair, and so on.
While it may never be feasible to compare such counterfactual states of the world, this does not mean that there are no experiential truths to be compared.
Once again, there is a difference between \textit{answers in practice and answers in principle}.

The moment one begins thinking about morality in terms of well-being, it becomes remarkably easy to discern a moral hierarchy across human societies.
Consider the following account of the Dobu islanders from Ruth Benedict :

Life in Dobu fosters extreme forms of animosity and malignancy which most societies have minimized by their institutions.
Dobuan institutions, on the other hand, exalt them to the highest degree.
The Dobuan lives out without repression man’s worst nightmares of the ill-will of the universe, and according to his view of life virtue consists in selecting a victim upon whom he can vent the malignancy he attributes alike to human society and to the powers of nature.
All existence appears to him as a cutthroat struggle in which deadly antagonists are pitted against one another in contest for each one of the goods of life.
Suspicion and cruelty are his trusted weapons in the strife and he gives no mercy, as he asks none.
The Dobu appear to have been as blind to the possibility of true cooperation as they were to the truths of modern science.
While innumerable things would have been worthy of their attention --- the Dobu were, after all, extremely poor and mightily ignorant --- their main preoccupation seems to have been malicious sorcery.
Every Dobuan’s primary interest was to cast spells on other members of the tribe in an effort to sicken or kill them and in the hopes of magically appropriating their crops.
The relevant spells were generally passed down from a maternal uncle and became every Dobuan’s most important possessions.
Needless to say, those who received no such inheritance were believed to be at a terrible disadvantage.
Spells could be purchased, however, and the economic life of the Dobu was almost entirely devoted to trade in these fantastical commodities.

Certain members of the tribe were understood to have a monopoly over both the causes and cures for specific illnesses.
Such people were greatly feared and ceaselessly propitiated.
In fact, the conscious application of magic was believed necessary for the most mundane tasks.
Even the work of gravity had to be supplemented by relentless wizardry :
absent the right spell, a man’s vegetables were expected to rise out of the soil and vanish under their own power.

To make matters worse, the Dobu imagined that good fortune conformed to a rigid law of thermodynamics :
if one man succeeded in growing more yams than his neighbor, his surplus crop must have been pilfered through sorcery.
As all Dobu continuously endeavored to steal one another’s crops by such methods, the lucky gardener is likely to have viewed his surplus in precisely these terms.
A good harvest, therefore, was tantamount to ``a confession of theft.''

This strange marriage of covetousness and magical thinking created a perfect obsession with secrecy in Dobu society.
Whatever possibility of love and real friendship remained seems to have been fully extinguished by a final doctrine :
the power of sorcery was believed to grow in proportion to one’s intimacy with the intended victim.
This belief gave every Dobuan an incandescent mistrust of all others, which burned brightest on those closest.
Therefore, if a man fell seriously ill or died, his misfortune was immediately blamed on his wife, and vice versa.
The picture is of a society completely in thrall to antisocial delusions.

Did the Dobu love their friends and family as much as we love ours?
Many people seem to think that the answer to such a question must, in principle, be ``yes,'' or that the question itself is vacuous.
I think it is clear, however, that the question is well posed and easily answered.
The answer is ``no.''
Being fellow Homo sapiens, we must presume that the Dobu islanders had brains sufficiently similar to our own to invite comparison.
Is there any doubt that the selfishness and general malevolence of the Dobu would have been expressed at the level of their brains?
Only if you think the brain does nothing more than filter oxygen and glucose out of the blood.
Once we more fully understand the neurophysiology of states like love, compassion, and trust, it will be possible to spell out the differences between ourselves and people like the Dobu in greater detail.
But we need not await any breakthroughs in neuroscience to bring the general principle in view :
just as it is possible for individuals and groups to be wrong about how best to maintain their physical health, it is possible for them to be wrong about how to maximize their personal and social well-being.

I believe that we will increasingly understand good and evil, right and wrong, in scientific terms, because moral concerns translate into facts about how our thoughts and behaviors affect the well-being of conscious creatures like ourselves.
If there are facts to be known about the well-being of such creatures --- and there are --- then there must be right and wrong answers to moral questions.
Students of philosophy will notice that this commits me to some form of moral realism (viz. moral claims can really be true or false) and some form of consequentialism (viz. the rightness of an act depends on how it impacts the well-being of conscious creatures).
While moral realism and consequentialism have both come under pressure in philosophical circles, they have the virtue of corresponding to many of our intuitions about how the world works.

Here is my (consequentialist) starting point :
all questions of value (right and wrong, good and evil, etc.) depend upon the possibility of experiencing such value.
Without potential consequences at the level of experience --- happiness, suffering, joy, despair, etc. --- all talk of value is empty.
Therefore, to say that an act is morally necessary, or evil, or blameless, is to make (tacit) claims about its consequences in the lives of conscious creatures (whether actual or potential).
I am unaware of any interesting exception to this rule.
Needless to say, if one is worried about pleasing God or His angels, this assumes that such invisible entities are conscious (in some sense) and cognizant of human behavior.
It also generally assumes that it is possible to suffer their wrath or enjoy their approval, either in this world or the world to come.
Even within religion, therefore, consequences and conscious states remain the foundation of all values.

Consider the thinking of a Muslim suicide bomber who decides to obliterate himself along with a crowd of infidels :
this would appear to be a perfect repudiation of the consequentialist attitude.
And yet, when we look at the rationale for seeking martyrdom within Islam, we see that the consequences of such actions, both real and imagined, are entirely the point.
Aspiring martyrs expect to please God and experience an eternity of happiness after death.
If one fully accepts the metaphysical presuppositions of traditional Islam, martyrdom must be viewed as the ultimate attempt at career advancement.
The martyr is also the greatest of altruists :
for not only does he secure a place for himself in Paradise, he wins admittance for seventy of his closest relatives as well.
Aspiring martyrs also believe that they are furthering God’s work here on earth, with desirable consequences for the living.
We know quite a lot about how such people think --- indeed, they advertise their views and intentions ceaselessly --- and it has everything to do with their belief that God has told them, in the Qur’an and the hadith, precisely what the consequences of certain thoughts and actions will be.
Of course, it seems profoundly unlikely that our universe has been designed to reward individual primates for killing one another while believing in the divine origin of a specific book.
The fact that would be martyrs are almost surely wrong about the consequences of their behavior is precisely what renders it such an astounding and immoral misuse of human life.

Because most religions conceive of morality as a matter of being obedient to the word of God (generally for the sake of receiving a supernatural reward), their precepts often have nothing to do with maximizing well-being in this world.
Religious believers can, therefore, assert the immorality of contraception, masturbation, homosexuality, etc., without ever feeling obliged to argue that these practices actually cause suffering.
They can also pursue aims that are flagrantly immoral, in that they needlessly perpetuate human misery, while believing that these actions are morally obligatory.
This pious uncoupling of moral concern from the reality of human and animal suffering has caused tremendous harm.

Clearly, there are mental states and capacities that contribute to our general well-being (happiness, compassion, kindness, etc.) as well as mental states and incapacities that diminish it (cruelty, hatred, terror, etc.).
It is, therefore, meaningful to ask whether a specific action or way of thinking will affect a person’s well-being and/or the well-being of others, and there is much that we might eventually learn about the biology of such effects.
Where a person finds himself on this continuum of possible states will be determined by many factors ---- genetic, environmental, social, cognitive, political, economic, etc. --- and while our understanding of such influences may never be complete, their effects are realized at the level of the human brain.
Our growing understanding of the brain, therefore, will have increasing relevance for any claims we make about how thoughts and actions affect the welfare of human beings.

Notice that I do not mention morality in the preceding paragraph, and perhaps I need not.
I began this book by arguing that, despite a century of timidity on the part of scientists and philosophers, morality can be linked directly to facts about the happiness and suffering of conscious creatures.
However, it is interesting to consider what would happen if we simply ignored this step and merely spoke about ``well-being.''
What would our world be like if we ceased to worry about ``right'' and ``wrong,'' or ``good'' and ``evil,'' and simply acted so as to maximize well-being, our own and that of others?
Would we lose anything important?
And if important, wouldn’t it be, by definition, a matter of someone’s well-being?

\subsection{Can We Ever Be "Right" About Right and Wrong?}

The philosopher and neuroscientist Joshua Greene has done some of the most influential neuroimaging research on morality.
While Greene wants to understand the brain processes that govern our moral lives, he believes that we should be skeptical of moral realism on metaphysical grounds.
For Greene, the question is not, ``How can you know for sure that your moral beliefs are true?'' but rather, ``How could it be that anyone’s moral beliefs are true?''
In other words, what is it about the world that could make a moral claim true or false?
He appears to believe that the answer to this question is ``nothing.''

However, it seems to me that this question is easily answered.
Moral view A is truer than moral view B, if A entails a more accurate understanding of the connections between human thoughts/intentions/behavior and human well-being.
Does forcing women and girls to wear burqas make a net positive contribution to human well-being?
Does it produce happier boys and girls?
Does it produce more compassionate men or more contented women?
Does it make for better relationships between men and women, between boys and their mothers, or between girls and their fathers?
I would bet my life that the answer to each of these questions is ``no.''
So, I think, would many scientists.
And yet, as we have seen, most scientists have been trained to think that such judgments are mere expressions of cultural bias --- and, thus, unscientific in principle.
Very few of us seem willing to admit that such simple, moral truths increasingly fall within the scope of our scientific worldview.
Greene articulates the prevailing skepticism quite well :

Moral judgment is, for the most part, driven not by moral reasoning, but by moral intuitions of an emotional nature.
Our capacity for moral judgment is a complex evolutionary adaptation to an intensely social life.
We are, in fact, so well adapted to making moral judgments that our making them is, from our point of view, rather easy, a part of ``common sense.''
And like many of our common sense abilities, our ability to make moral judgments feels to us like a perceptual ability, an ability, in this case, to discern immediately and reliably mind-independent moral facts.
As a result, we are naturally inclined toward a mistaken belief in moral realism.
The psychological tendencies that encourage this false belief serve an important biological purpose, and that explains why we should find moral realism so attractive even though it is false.
Moral realism is, once again, a mistake we were born to make.

Greene alleges that moral realism assumes that ``there is sufficient uniformity in people’s underlying moral outlooks to warrant speaking as if there is a fact of the matter about what’s `right' or `wrong,' `just' or `unjust.' ''
But do we really need to assume such uniformity for there to be right answers to moral questions?
Is physical or biological realism predicated on ``sufficient uniformity in people’s underlying [physical or biological] outlooks''?
Taking humanity as a whole, I am quite certain that there is a greater consensus that cruelty is wrong (a common moral precept) than the passage of time varies with velocity (special relativity) or that humans and lobsters share a common ancestor (evolution).
Should we doubt whether there is a ``fact of the matter'' with respect to these physical and biological truth claims?
Does the general ignorance about the special theory of relativity or the pervasive disinclination of Americans to accept the scientific consensus on evolution put our scientific worldview, even slightly, in question?

Greene notes that it is often difficult to get people to agree about moral truth, or to even get an individual to agree with himself in different contexts.
These tensions lead him to the following conclusion :

[M]oral theorizing fails because our intuitions do not reflect a coherent set of moral truths and were not designed by natural selection or anything else to behave as if they were \dots
If you want to make sense of your moral sense, turn to biology, psychology, and sociology --- not normative ethics.

This objection to moral realism may seem reasonable, until one notices that it can be applied, with the same leveling effect, to any domain of human knowledge.
For instance, it is just as true to say that our logical, mathematical, and physical intuitions have not been designed by natural selection to track the Truth.
Does this mean that we must cease to be realists with respect to physical reality?
We need not look far in science to find ideas and opinions that defy easy synthesis.
There are many scientific frameworks (and levels of description) that resist integration and which divide our discourse into areas of specialization, even pitting Nobel laureates in the same discipline against one another.
Does this mean that we can never hope to understand what is really going on in the world?
No.
It means the conversation must continue.

Total uniformity in the moral sphere --- either interpersonally or intrapersonally --- may be hopeless.
So what?
This is precisely the lack of closure we face in all areas of human knowledge.
Full consensus as a scientific goal only exists in the limit, at a hypothetical end of inquiry.
Why not tolerate the same open-endedness in our thinking about human well-being?

Again, this does not mean that all opinions about morality are justified.
To the contrary --- the moment we accept that there are right and wrong answers to questions of human well-being, we must admit that many people are simply wrong about morality.
The eunuchs who tended the royal family in China’s Forbidden City, dynasty after dynasty, seem to have felt generally well compensated for their lives of arrested development and isolation by the influence they achieved at court --- as well as by the knowledge that their genitalia, which had been preserved in jars all the while, would be buried with them after their deaths, ensuring them rebirth as human beings.
When confronted with such an exotic point of view, a moral realist would like to say we are witnessing more than a mere difference of opinion :
we are in the presence of moral error.
It seems to me that we can be reasonably confident that it is bad for parents to sell their sons into the service of a government that intends to cut off their genitalia ``using only hot chili sauce as a local anesthetic.''
This would mean that Sun Yaoting, the emperor’s last eunuch, who died in 1996 at the age of ninety-four, was wrong to harbor, as his greatest regret, ``the fall of the imperial system he had aspired to serve.''
Most scientists seem to believe that no matter how maladaptive or masochistic a person’s moral commitments, it is impossible to say that he is ever mistaken about what constitutes a good life.

\subsection{Moral Paradox}

One of the problems with consequentialism in practice is that we cannot always determine whether the effects of an action will be bad or good.
In fact, it can be surprisingly difficult to decide this even in retrospect.
Dennett has dubbed this problem ``the Three Mile Island Effect.''
Was the meltdown at Three Mile Island a bad outcome or a good one?
At first glance, it surely seems bad, but it might have also put us on a path toward greater nuclear safety, thereby saving many lives.
Or it might have caused us to grow dependent on more polluting technologies, contributing to higher rates of cancer and to global climate change.
Or it might have produced a multitude of effects, some mutually reinforcing, and some mutually canceling.
If we cannot determine the net result of even such a well-analyzed event, how can we judge the likely consequences of the countless decisions we must make throughout our lives?

One difficulty we face in determining the moral valence of an event is that it often seems impossible to determine whose well-being should most concern us.
People have competing interests, mutually incompatible notions of happiness, and there are many well-known paradoxes that leap into our path the moment we begin thinking about the welfare of whole populations.
As we are about to see, population ethics is a notorious engine of paradox, and no one, to my knowledge, has come up with a way of assessing collective well-being that conserves all of our intuitions.
As the philosopher Patricia Churchland puts it, ``no one has the slightest idea how to compare the mild headache of five million against the broken legs of two, or the needs of one’s own two children against the needs of a hundred unrelated brain-damaged children in Serbia.''
Such puzzles may seem of mere academic interest, until we realize that population ethics governs the most important decisions societies ever make.
What are our moral responsibilities in times of war, when diseases spread, when millions suffer famine, or when global resources are scarce?
These are moments in which we have to assess changes in collective welfare in ways that purport to be rational and ethical.
Just how motivated should we be to act when 250,000 people die in an earthquake on the island of Haiti?
Whether we know it or not, intuitions about the welfare of whole populations determine our thinking on these matters.
Except, that is, when we simply ignore population ethics --- as, it seems, we are psychologically disposed to do.
The work of the psychologist Paul Slovic and colleagues has uncovered some rather startling limitations on our capacity for moral reasoning when thinking about large groups of people --- or, indeed, about groups larger than one.
As Slovic observes, when human life is threatened, it seems both rational and moral for our concern to increase with the number of lives at stake.
And if we think that losing many lives might have some additional negative consequences (like the collapse of civilization), the curve of our concern should grow steeper still.
But this is not how we characteristically respond to the suffering of other human beings.

Slovic’s experimental work suggests that we intuitively care most about a single, identifiable human life, less about two, and we grow more callous as the body count rises.
Slovic believes that this ``psychic numbing'' explains the widely lamented fact that we are generally more distressed by the suffering of single child (or even a single animal) than by a proper genocide.
What Slovic has termed ``genocide neglect'' --- our reliable failure to respond, both practically and emotionally, to the most horrific instances of unnecessary human suffering --- represents one of the more perplexing and consequential failures of our moral intuition.
Slovic found that when given a chance to donate money in support of needy children, subjects give most generously and feel the greatest empathy when told only about a single child’s suffering.
When presented with two needy cases, their compassion wanes.
And this diabolical trend continues :
the greater the need, the less people are emotionally affected and the less they are inclined to give.

Of course, charities have long understood that putting a face on the data will connect their constituents to the reality of human suffering and increase donations.
Slovic’s work has confirmed this suspicion, which is now known as the ``identifiable victim effect.''
Amazingly, however, adding information about the scope of a problem to these personal appeals proves to be counterproductive.
Slovic has shown that setting the story of a single needy person in the context of wider human need reliably diminishes altruism.

The fact that people seem to be reliably less concerned when faced with an increase in human suffering represents an obvious violation of moral norms.
The important point, however, is that we immediately recognize how indefensible this allocation of emotional and material resources is once it is brought to our attention.
What makes these experimental findings so striking is that they are patently inconsistent :
if you care about what happens to one little girl, and you care about what happens to her brother, you must, at the very least, care as much about their combined fate.
Your concern should be (in some sense) cumulative.
When your violation of this principle is revealed, you will feel that you have committed a moral error.
This explains why results of this kind can only be obtained between subjects (where one group is asked to donate to help one child and another group is asked to support two);
we can be sure that if we presented both questions to each participant in the study, the effect would disappear (unless subjects could be prevented from noticing when they were violating the norms of moral reasoning).

Clearly, one of the great tasks of civilization is to create cultural mechanisms that protect us from the moment-to-moment failures of our ethical intuitions.
We must build our better selves into our laws, tax codes, and institutions.
Knowing that we are generally incapable of valuing two children more than either child alone, we must build a structure that reflects and enforces our deeper understanding of human well-being.
This is where a science of morality could be indispensable to us :
the more we understand the causes and constituents of human fulfillment, and the more we know about the experiences of our fellow human beings, the more we will be able to make intelligent decisions about which social policies to adopt.
For instance, there are an estimated 90,000 people living on the streets of Los Angeles.
Why are they homeless?
How many of these people are mentally ill?
How many are addicted to drugs or alcohol?
How many have simply fallen through the cracks in our economy?
Such questions have answers.
And each of these problems admits of a range of responses, as well as false solutions and neglect.
Are there policies we could adopt that would make it easy for every person in the United States to help alleviate the problem of homelessness in their own communities?
Is there some brilliant idea that no one has thought of that would make people want to alleviate the problem of homelessness more than they want to watch television or play video games?
Would it be possible to design a video game that could help solve the problem of homelessness in the real world?
Again, such questions open onto a world of facts, whether or not we can bring the relevant facts into view.

Clearly, morality is shaped by cultural norms to a great degree, and it can be difficult to do what one believes to be right on one’s own.
A friend’s four-year-old daughter recently observed the role that social support plays in making moral decisions :
``It’s so sad to eat baby lambies,'' she said as she gnawed greedily on a lamb chop.
``So, why don’t you stop eating them?'' her father asked.
``Why would they kill such a soft animal?
Why wouldn’t they kill some other kind of animal?''
``Because,'' her father said, ``people like to eat the meat. Like you are, right now.''
His daughter reflected for a moment --- still chewing her lamb --- and then replied :
``It’s not good.
But I can’t stop eating them if they keeping killing them.''
And the practical difficulties for consequentialism do not end here.
When thinking about maximizing the well-being of a population, are we thinking in terms of total or average well-being?
The philosopher Derek Parfit has shown that both bases of calculation lead to troubling paradoxes.
If we are concerned only about total welfare, we should prefer a world with hundreds of billions of people whose lives are just barely worth living to a world in which 7 billion of us live in perfect ecstasy.
This is the result of Parfit’s famous argument known as ``The Repugnant Conclusion.''
If, on the other hand, we are concerned about the average welfare of a population, we should prefer a world containing a single, happy inhabitant to a world of billions who are only slightly less happy;
it would even suggest that we might want to painlessly kill many of the least happy people currently alive, thereby increasing the average of human well-being.
Privileging average welfare would also lead us to prefer a world in which billions live under the misery of constant torture to a world in which only one person is tortured ever-so-slightly more.
It could also render the morality of an action dependent upon the experience of unaffected people.
As Parfit points out, if we care about the average over time, we might deem it morally wrong to have a child today whose life, while eminently worth living, would not compare favorably to the lives of the ancient Egyptians.
Parfit has even devised scenarios in which everyone alive could have a lower quality of life than they otherwise would and yet the average quality of life will have increased.
Clearly, this proves that we cannot rely on a simple summation or averaging of welfare as our only metric.
And yet, at the extremes, we can see that human welfare must aggregate in some way :
it really is better for all of us to be deeply fulfilled than it is for everyone to live in absolute agony.

Placing only consequences in our moral balance also leads to indelicate questions.
For instance, do we have a moral obligation to come to the aid of wealthy, healthy, and intelligent hostages before poor, sickly, and slow-witted ones?
After all, the former are more likely to make a positive contribution to society upon their release.
And what about remaining partial to one’s friends and family?
Is it wrong for me to save the life of my only child if, in the process, I neglect to save a stranger’s brood of eight?
Wrestling with such questions has convinced many people that morality does not obey the simple laws of arithmetic.
However, such puzzles merely suggest that certain moral questions could be difficult or impossible to answer in practice;
they do not suggest that morality depends upon something other than the consequences of our actions and intentions.
This is a frequent source of confusion :
consequentialism is less a method of answering moral questions than it is a claim about the status of moral truth.
Our assessment of consequences in the moral domain must proceed as it does in all others :
under the shadow of uncertainty, guided by theory, data, and honest conversation.
The fact that it may often be difficult, or even impossible, to know what the consequences of our thoughts and actions will be does not mean that there is some other basis for human values that is worth worrying about.

Such difficulties notwithstanding, it seems to me quite possible that we will one day resolve moral questions that are often thought to be unanswerable.
For instance, we might agree that having a preference for one’s intimates is better (in that it increases general welfare) than being fully disinterested as to how consequences accrue.
Which is to say that there may be some forms of love and happiness that are best served by each of us being specially connected to a subset of humanity.
This certainly appears to be descriptively true of us at present.
Communal experiments that ignore parents’ special attachment to their own children, for instance, do not seem to work very well.
The Israeli kibbutzim learned this the hard way :
after discovering that raising children communally made both parents and children less happy, they reinstated the nuclear family.
Most people may be happier in a world in which a natural bias toward one’s own children is conserved --- presumably in the context of laws and social norms that disregard this bias.
When I take my daughter to the hospital, I am naturally more concerned about her than I am about the other children in the lobby.
I do not, however, expect the hospital staff to share my bias.
In fact, given time to reflect about it, I realize that I would not want them to.
How could such a denial of my self-interest actually be in the service of my self-interest?
Well, first, there are many more ways for a system to be biased against me than in my favor, and I know that I will benefit from a fair system far more than I will from one that can be easily corrupted.
I also happen to care about other people, and this experience of empathy deeply matters to me.
I feel better as a person valuing fairness, and I want my daughter to become a person who shares this value.
And how would I feel if the physician attending my daughter actually shared my bias for her and viewed her as far more important than the other patients under his care?
Frankly, it would give me the creeps.

But perhaps there are two possible worlds that maximize the well-being of their inhabitants to precisely the same degree :
in world X everyone is focused on the welfare of all others without bias, while in world Y everyone shows some degree of moral preference for their friends and family.
Perhaps these worlds are equally good, in that their inhabitants enjoy precisely the same level of well-being.
These could be thought of as two peaks on the moral landscape.
Perhaps there are others.
Does this pose a threat to moral realism or to consequentialism?
No, because there would still be right and wrong ways to move from our current position on the moral landscape toward one peak or the other, and movement would still be a matter of increasing well-being in the end.
To bring the discussion back to the especially low-hanging fruit of conservative Islam :
there is absolutely no reason to think that demonizing homosexuals, stoning adulterers, veiling women, soliciting the murder of artists and intellectuals, and celebrating the exploits of suicide bombers will move humanity toward a peak on the moral landscape.
This is, I think, as objective a claim as we ever make in science.
Consider the Danish cartoon controversy :
an eruption of religious insanity that still flows to this day.
Kurt Westergaard, the cartoonist who drew what was arguably the most inflammatory of these utterly benign cartoons has lived in hiding since pious Muslims first began calling for his murder in 2006.
A few weeks ago --- more than three years after the controversy first began --- a Somali man broke into Westergaard’s home with an axe.
Only the construction of a specially designed ``safe room'' allowed Westergaard to escape being slaughtered for the glory of God (his five-year-old granddaughter also witnessed the attack).
Westergaard now lives with continuous police protection --- as do the other eighty-seven men in Denmark who have the misfortune of being named ``Kurt Westergaard.''
The peculiar concerns of Islam have created communities in almost every society on earth that grow so unhinged in the face of criticism that they will reliably riot, burn embassies, and seek to kill peaceful people, over cartoons.
This is something they will not do, incidentally, in protest over the continuous atrocities committed against them by their fellow Muslims.
The reasons why such a terrifying inversion of priorities does not tend to maximize human happiness are susceptible to many levels of analysis --- ranging from biochemistry to economics.
But do we need further information in this case?
It seems to me that we already know enough about the human condition to know that killing cartoonists for blasphemy does not lead anywhere worth going on the moral landscape.
There are other results in psychology and behavioral economics that make it difficult to assess changes in human well-being.
For instance, people tend to consider losses to be far more significant than forsaken gains, even when the net result is the same.
For instance, when presented with a wager where they stand a 50 percent chance of losing \$100, most people will consider anything less than a potential gain of \$200 to be unattractive.
This bias relates to what has come to be known as ``the endowment effect'' :
people demand more money in exchange for an object that has been given to them than they would spend to acquire the object in the first place.
In psychologist Daniel Kahneman’s words, ``a good is worth more when it is considered as something that could be lost or given up than when it is evaluated as a potential gain.''
This aversion to loss causes human beings to generally err on the side of maintaining the status quo.
It is also an important impediment to conflict resolution through negotiation :
for if each party values his opponent’s concessions as gains and his own as losses, each is bound to perceive his sacrifice as being greater.
Loss aversion has been studied with functional magnetic resonance imaging (fMRI).
If this bias were the result of negative feelings associated with potential loss, we would expect brain regions known to govern negative emotion to be involved.
However, researchers have not found increased activity in any areas of the brain as losses increase.
Instead, those regions that represent gains show decreasing activity as the size of the potential losses increases.
In fact, these brain structures themselves exhibit a pattern of ``neural loss aversion'' :
their activity decreases at a steeper rate in the face of potential losses than they increase for potential gains.
There are clearly cases in which such biases seem to produce moral illusions --- where a person’s view of right and wrong will depend on whether an outcome is described in terms of gains or losses.
Some of these illusions might not be susceptible to full correction.
As with many perceptual illusions, it may be impossible to ``see'' two circumstances as morally equivalent, even while ``knowing'' that they are.
In such cases, it may be ethical to ignore how things seem.
Or it may be that the path we take to arrive at identical outcomes really does matter to us --- and, therefore, that losses and gains will remain incommensurable.

Imagine, for instance, that you are empaneled as the member of a jury in a civil trial and asked to determine how much a hospital should pay in damages to the parents of children who received substandard care in their facility. 
There are two scenarios to consider : 
Couple A learned that their three-year-old daughter was inadvertently given a neurotoxin by the hospital staff. 
Before being admitted, their daughter was a musical prodigy with an IQ of 195. 
She has since lost all her intellectual gifts. 
She can no longer play music with any facility and her IQ is now a perfectly average 100. 
Couple B learned that the hospital neglected to give their three-year-old daughter, who has an IQ of 100, a perfectly safe and inexpensive genetic enhancement that would have given her remarkable musical talent and nearly doubled her IQ. 
Their daughter’s intelligence remains average, and she lacks any noticeable musical gifts. 
The critical period for giving this enhancement has passed. 
Obviously the end result under either scenario is the same. 
But what if the mental suffering associated with loss is simply bound to be greater than that associated with forsaken gains? 
If so, it may be appropriate to take this difference into account, even when we cannot give a rational explanation of why it is worse to lose something than not to gain it. 
This is another source of difficulty in the moral domain : 
unlike dilemmas in behavioral economics, it is often difficult to establish the criteria by which two outcomes can be judged equivalent.
There is probably another principle at work in this example, however : 
people tend to view sins of commission more harshly than sins of omission. 
It is not clear how we should account for this bias either.
 But, once again, to say that there are right answers to questions of how to maximize human well-being is not to say that we will always be in a position to answer such questions. 
 There will be peaks and valleys on the moral landscape, and movement between them is clearly possible, whether or not we always know which way is up.  
 
 There are many other features of our subjectivity that have implications for morality. 
 For instance, people tend to evaluate an experience based on its peak intensity (whether positive or negative) and the quality of its final moments. 
 In psychology, this is known as the ``peak/end rule.'' 
 Testing this rule in a clinical environment, one group found that patients undergoing colonoscopies (in the days when this procedure was done without anesthetic) could have their perception of suffering markedly reduced, and their likelihood of returning for a follow-up exam increased, if their physician needlessly prolonged the procedure at its lowest level of discomfort by leaving the colonoscope inserted for a few extra minutes.
 The same principle seems to hold for aversive sounds and for exposure to cold. 
 Such findings suggest that, under certain conditions, it is compassionate to prolong a person’s pain unnecessarily so as to reduce his memory of suffering later on. 
 Indeed, it might be unethical to do otherwise. 
 Needless to say, this is a profoundly counterintuitive result. 
 But this is precisely what is so important about science : 
 it allows us to investigate the world, and our place within it, in ways that get behind first appearances. 
 Why shouldn’t we do this with morality and human values generally?


\newpage
\section{Chapter 3 : Belief}

\newpage
\section{Chapter 4 : Religion}

\newpage
\section{Chapter 5 : The Future of Happiness}

\newpage
\section{Acknowledgements}

\newpage
\section{Notes}

\newpage
\section{References}

\newpage
\section{Index}

\end{document}
