\documentclass[a4paper,12pt]{extbook}
\usepackage[utf8]{inputenc}

\title{The Moral Landscape : How Science Can Determine Human Values}
\author{Sam Harris}

\begin{document}

\maketitle
\tableofcontents

\newpage
\section{Introduction}

\subsection{The Moral Landscape}

The people of Albania have a venerable tradition of vendetta called \textit{Kanun} :
if a man commits a murder, his victim's family can kill any one of his male relatives in reprisal.
If a boy has the misfortune of being the son or brother of a murderer, he must spend his days and nights in hiding, forgoing a proper education, adequate health care, and the pleasures of a normal life.
Untold numbers of Albanian men and boys live as prisoners of their homes even now.
Can we say that the Albanians are morally wrong to have structured their society in this way?
Is their tradition of blood feud a form of evil?
Are their values inferior to our own?

Most people imagine that science cannot pose, much less answer, questions of this sort.
How could we ever say, as a matter of scientific fact, that one way of life is better, or moral, than another?
Whose defintion of ``better'' or ``moral'' would we use?
While many scientists now study the evolution of morality, as well as its underlying neurobiology, the purpose of their research is merely to describe how human beings think and behave.
No one expects science to tell us how we \textit{ought} to think and behave.
Controversies about human values are controversies about which science officially has no opinion.

I will argue, however, that questions about values --- about meaning, morality, and life's larger purpose --- are really questions about the well-being of conscious creatures.
Values, therefore, translate into facts that can be scientifically understood: regarding positive and negative social emotions, retributive impulses, the effects of specific laws and social institutions on human relationships, the neurophysiology of happiness and suffering, etc.
The most important of these facts are bound to transcend culture --- just as facts about physical and mental health do.
Cancer in the highlands of New Guinea is still cancer;
cholera is still cholea;
schizophrenia is still schizophrenia;
and so, too, I will argue, compassion is still compassion, and well-being is still well-being.
And if these are important cultural differences in how people flourish --- if, for instance, there are incompatible but equivalent ways to raise happy, intelligent, and creative children --- these differences are also facts that must depend upon the organization of the human brain.
In principle, therefore, we can account for the ways in which culture defines us within the context of neuroscience and psychology.
The more we understand ourselves at the level of the brain, the more we will see that there are right and wrong answers to questions of human values.

Of course, we will have to confront some ancient disagreements about the status of moral truth:
people who draw their worldview from religion generally believe that moral truths exists, but only because God has woven it into the very fabric of reality;
while those who lack such faith tend to think that notions of ``good'' and ``evil'' must be the products of evolutionary pressure and cultural invention.
On the first account, to speak of ``moral truth'' is, of necessity, to invoke God;
on the second, it is merely to give voice to one's apish urges, cultural biases, and philosophical confusion.
My purpose is to persuade you that both sides in this debate are wrong.
The goal of this book is to begin a convesation about how moral truth can be understood in the context of science.

While the argument I make in this book is bound to be controvversial, it rests on a very simple premise :
human well-being entirely depends in the events in the world and on states of human brain.
Consequently, there must be scientific truths to be known about it.
A more detailed understanding of these truths will force us to draw clear distinctions between different ways of living in society with one another, judging some to be better or worse, more or less true to the facts, and more or less ethical.
Clearly, such insights could help us improve the quality of human life --- and this is where academic debate ends and choices affecting the lives of millions of people begin.

I am not suggesting that we are guaranteed to resolve every moral controversy through science.
Differences of opinion will remain --- but opinions will be increasingly constraints by facts.
And it is important to realize that our inability to answer a question say nothing whther the question itself has an answer.
Exactly how many people were bitten by mosquitoes in the last sixty seconds?
How many of these people will contract malaria?
How many will die as a result?
Given the technical challenges involved, no team of scientists could possibly respond to such questions.
And yet we know that they admit of simple numerical answers.
Does our inablity to gather the relevant data oblige us to respect all opinions equally?
Of course not,
In the same way, the fact that we not be able to resolve specific moral dilemmas does not suggest that all competing responses to them are equally valid.
In my experience, mistaking \textit{no answers in practice} for \textit{fno answers in principle} is a great source of moral confusion.

There are, for instance, twenty-one U. S. states that still allow corporal punishment in their schools.
These are places where it is actually legal for a teacher to beat a child with a wooden board hard enough to raise large bruises and even to break the skin.
Hundreds of thousands of children are subjected to this violence each year, almost exclusively in the South.
Needless to say, the rationale for this behavior is explicitly religious :
for the Creator of the Universe Himself has told us not to spare the rod, lest we spoil the Child (Proverbs 13:24, 20:30 and 23:13-14).
However, if we are actually concerned about human well-being, and would treat children in such a way as to promote it, we might wonder whether it is generally wise to subject little boys and girls to pain, terror, and public humiliation as a means of encouraging their cognitive and emotional development.
Is there any doubt that this question \textit{has} an answer?
Is there any doubt that it matters that we get it right?
In fact, all the research indicates that corporal punishment is a disastrous practice, leading to more violence and social pathology --- and, perversely, to greater support for corporal punishment.

But the deeper point is that there simply must be answers to questions of this kind, whether we know them or not.
And these are not areas where we can afford to simply respect the ``traditions'' of others and agree to disagree.
Why will science increasingly decide such questions?
Because the discrepant answers people give to them --- along with the consequences that follow in terms of human relationships, states of mind, acts of violence, entanglements with the law, etc. --- translate into differences in our brains, in the brains of others, and in the world at large.
I hope to show that when talking about values, we are actually talking about an interdependant worlds of facts.

There are facts to be understood about how thoughts and intentions arise in the human brain;
there are facts to be learned about how these mental states translate into behavior;
there are further facts to be known about how these behaviors influence the world and the experience of other conscious beings.
We will see that facts of this sort exhaust what we can reasonably mean by terms like ``good'' and ``evil''.
They will also increasingly fall within the purview of science and run far deeeper than a person's religious affiliation.
Just as there is no such things as Christian physics or Muslim algebra, we will see that there is no such things as Christian or Muslim morality.
Indeed, I will argue that morality should be considered an underdeveloped branch of science.

Since the publication of my first book, \textit{The End of Faith}, I have had a privileged view of the ``culture wars'' --- both in the United States, between secular liberals and Christian conservatives, and in Europe, between largely irreligious societies and their growing Muslim populations.
Having received tens of thousands of letters and emails from people at every pint on the continuum between faith and doubt, I can say with some confidence that a shared belief in the limitations of reason lies at the bottom these cultural divides.
Both sides believe that reason is powerless to answer the most important questions in human life.
And how a person perceives the gulf between facts and values seems to influence his views on almost every issue of social importance --- from the fighting of wars to the education of children.

This rupture in our thinking has different consequences at each end of the political spectrum :
religious conversatives tend to believe that there are right answers to questions of meaning and morality, but only because the God of Abraham deems it so.
They concede that ordinary facts can be discovered through rational inquiry, but they believe that values must come from a voice in a whirlwind.
Scriptural literalism, intolerance of diversity, mistrust of science, disregard for the real causes of human and animal suffering --- too often, this is how the division between facts and values expresses itself on the religious right.

Secular liberals, on the other hand, tend to imagine that no objectives answers to moral questions exist.
While John Stuart Mill might conform to \textit{our} cultural ideal of goodness better than Osama Bin Laden does, most secularists suspect that Mill's ideas about right and wrong reach no closer to the Truth.
Multiculturalism, moral relativism, political correctness, tolerance even of \textit{intolerance} --- these are familiar consequences of separating facts and values on the left.

It should concern us that these two orientations are not equally empowering.
Increasingly, secular democracies are left supine before the unreasoning zeal of old-time religion.
The juxtaposition of conservative dogmatism and liberal doubt accounts for the decade that has been lost in the United States to a ban on federal funding for embryonic stem-cell research;
it explains the years of political distraction we have suffered, and will continue to suffer, over issues like abortion and gay marriage;
it lies at the bottom of current efforts to pass antiblasphemy laws at the United Nations (which would make it illegal for the citizens of member states to criticize religion);
it has hobbled the West in its generational war against radical Islam;
and it may yet refashion the societies of Europe into a new Caliphate.
Knowing what the Creator of the Universe believes about right and wrong inspires religious conservatives to enforce this vision in the public sphere at almost any cost; not knowing what is right—or that anything can ever be truly right—often leads secular liberals to surrender their intellectual standards and political freedoms with both hands.

The scientific community is predominantly secular and liberal --- and the concessions that scientists have made to religious dogmatism have been breathtaking.
As we will see, the problem reaches as high as the National Academies of Science and the National Institutes of Health.
Even the journal \textit{Nature}, the most influential scientific publication on Earth, has been unable to reliably police the boundary between reasoned discourse and pious fiction.
I recently reviewed every appearance of the term ``religion'' in the journal going back ten years and found that \textit{Nature}’s editors have generally accepted Stephen J. Gould’s doomed notion of ``nonoverlapping magisteria'' --- the idea that science and religion, properly construed, cannot be in conflict because they constitute different domains of expertise.
As one editorial put it, problems arise only when these disciplines ``stray onto each other’s territories and stir up trouble. ''
The underlying claim is that while science is the best authority on the workings of the physical universe, religion is the best authority on meaning, values, morality, and the good life.
I hope to persuade you that this is not only untrue, it could not possibly be true.
Meaning, values, morality, and the good life must relate to facts about the well-being of conscious creatures --- and, in our case, must lawfully depend upon events in the world and upon states of the human brain.
Rational, open-ended, honest inquiry has always been the true source of insight into such processes.
Faith, if it is ever right about anything, is right by accident.

The scientific community’s reluctance to take a stand on moral issues has come at a price.
It has made science appear divorced, in principle, from the most important questions of human life.
From the point of view of popular culture, science often seems like little more than a hatchery for technology.
While most educated people will concede that the scientific method has delivered centuries of fresh embarrassment to religion on matters of fact, it is now an article of almost unquestioned certainty, both inside and outside scientific circles, that science has nothing to say about what constitutes a good life.
Religious thinkers in all faiths, and on both ends of the political spectrum, are united on precisely this point;
the defense one most often hears for belief in God is not that there is compelling evidence for His existence, but that faith in Him is the only reliable source of meaning and moral guidance.
Mutually incompatible religious traditions now take refuge behind the same non sequitur.

It seems inevitable, however, that science will gradually encompass life’s deepest questions --- and this is guaranteed to provoke a backlash.
How we respond to the resulting collision of worldviews will influence the progress of science, of course, but it may also determine whether we succeed in building a global civilization based on shared values.
The question of how human beings should live in the twenty-first century has many competing answers --- and most of them are surely wrong.
Only a rational understanding of human well-being will allow billions of us to coexist peacefully, converging on the same social, political, economic, and environmental goals.
A science of human flourishing may seem a long way off, but to achieve it, we must first acknowledge that the intellectual terrain actually exists.

Throughout this book I make reference to a hypothetical space that I call ``the moral landscape'' --- a space of real and potential outcomes whose peaks correspond to the heights of potential well-being and whose valleys represent the deepest possible suffering.
Different ways of thinking and behaving --- different cultural practices, ethical codes, modes of government, etc. --- will translate into movements across this landscape and, therefore, into different degrees of human flourishing.
I’m not suggesting that we will necessarily discover one right answer to every moral question or a single best way for human beings to live.
Some questions may admit of many answers, each more or less equivalent.
However, the existence of multiple peaks on the moral landscape does not make them any less real or worthy of discovery.
Nor would it make the difference between being on a peak and being stuck deep in a valley any less clear or consequential.

To see that multiple answers to moral questions need not pose a problem for us, consider how we currently think about food :
no one would argue that there must be oneright food to eat.
And yet there is still an objective difference between healthy food and poison.
There are exceptions --- some people will die if they eat peanuts, for instance --- but we can account for these within the context of a rational discussion about chemistry, biology, and human health.
The world’s profusion of foods never tempts us to say that there are no facts to be known about human nutrition or that all culinary styles must be equally healthy in principle.

Movement across the moral landscape can be analyzed on many levels --- ranging from biochemistry to economics --- but where human beings are concerned, change will necessarily depend upon states and capacities of the human brain.
While I fully support the notion of ``consilience'' in science --- and, therefore, view the boundaries between scientific specialties as primarily a function of university architecture and limitations on how much any one person can learn in a lifetime --- the primacy of neuroscience and the other sciences of mind on questions of human experience cannot be denied.
Human experience shows every sign of being determined by, and realized in, states of the human brain.

Many people seem to think that a universal conception of morality requires that we find moral principles that admit of no exceptions.
If, for instance, it is truly wrong to lie, it must \textit{always} be wrong to lie --- and if one can find a single exception, any notion of moral truth must be abandoned.
But the existence of moral truth --- that is, the connection between how we think and behave and our well-being --- does not require that we define morality in terms of unvarying moral precepts.
Morality could be a lot like chess :
there are surely principles that generally apply, but they might admit of important exceptions.
If you want to play good chess, a principle like ``Don’t lose your Queen'' is almost always worth following.
But it admits of exceptions :
sometimes sacrificing your Queen is a brilliant thing to do;
occasionally, it is the \textit{only} thing you can do.
It remains a fact, however, that from any position in a game of chess there will be a range of objectively good moves and objectively bad ones.
If there are objective truths to be known about human well-being --- if kindness, for instance, is generally more conducive to happiness than cruelty is --- then science should one day be able to make very precise claims about which of our behaviors and uses of attention are morally good, which are neutral, and which are worth abandoning.

While it is too early to say that we have a full understanding of how human beings flourish, a piecemeal account is emerging.
Consider, for instance, the connection between early childhood experience, emotional bonding, and a person’s ability to form healthy relationships later in laife.
We know, of course, that emotional neglect and abuse are not good for us, psychologically or socially.
We also know that the effects of early childhood experience must be realized in the brain.
Research on rodents suggests that parental care, social attachment, and stress regulation are governed, in part, by the hormones vasopressin and oxytocin, because they influence activity in the brain’s reward system.
When asking why early childhood neglect is harmful to our psychological and social development, it seems reasonable to think that it might result from a disturbance in this same system.

While it would be unethical to deprive young children of normal care for the purposes of experiment, society inadvertently performs such experiments every day.
To study the effects of emotional deprivation in early childhood, one group of researchers measured the blood concentrations of oxytocin and vasopressin in two populations :
children raised in traditional homes and children who spent their first years in an orphanage.
As you might expect, children raised by the State generally do not receive normal levels of nurturing.
They also tend to have social and emotional difficulties later in life.
As predicted, these children failed to show a normal surge of oxytocin and vasopressin in response to physical contact with their adoptive mothers.
The relevant neuroscience is in its infancy, but we know that our emotions, social interactions, and moral intuitions mutually influence one another.
We grow attuned to our fellow human beings through these systems, creating culture in the process.
Culture becomes a mechanism for further social, emotional, and moral development.
There is simply no doubt that the human brain is the nexus of these influences.
Cultural norms influence our thinking and behavior by altering the structure and function of our brains.
Do you feel that sons are more desirable than daughters?
Is obedience to parental authority more important than honest inquiry?
Would you cease to love your child if you learned that he or she was gay?
The ways parents view such questions, and the subsequent effects in the lives of their children, must translate into facts about their brains.

My goal is to convince you that human knowledge and human values can no longer be kept apart.
The world of measurement and the world of meaning must eventually be reconciled.
And science and religion --- being antithetical ways of thinking about the same reality -- will never come to terms.
As with all matters of fact, differences of opinion on moral questions merely reveal the incompleteness of our knowledge;
they do not oblige us to respect a diversity of views indefinitely.

\subsection{Facts and Values}

The eighteenth-century Scottish philosopher David Hume famously argued that no description of the way the world is (facts) can tell us how we ought to behave (morality).
Following Hume, the philosopher G. E. Moore declared that any attempt to locate moral truths in the natural world was to commit a ``naturalistic fallacy.''
Moore argued that goodness could not be equated with any property of human experience (e.g., pleasure, happiness, evolutionary fitness) because it would always be appropriate to ask whether the property on offer was itself \textit{good}.
If, for instance, we were to say that goodness is synonymous with whatever gives people pleasure, it would still be possible to worry whether a specific instance of pleasure is actually \textit{good}.
This is known as Moore’s ``open question argument.''
And while I think this verbal trap is easily avoided when we focus on human well-being, most scientists and public intellectuals appear to have fallen into it.
Other influential philosophers, including Karl Popper, have echoed Hume and Moore on this point, and the effect has been to create a firewall between facts and values throughout our intellectual discourse.

While psychologists and neuroscientists now routinely study human happiness, positive emotions, and moral reasoning, they rarely draw conclusions about how human beings ought to think or behave in light of their findings.
In fact, it seems to be generally considered intellectually disreputable, even vaguely authoritarian, for a scientist to suggest that his or her work offers some guidance about how people should live.
The philosopher and psychologist Jerry Fodor crystallizes the view :

Science is about facts, not norms;
it might tell us how we are, but it couldn’t tell us what is wrong with how we are.
There couldn’t be a science of the human condition.

While it is rarely stated this clearly, this faith in the intrinsic limits of reason is now the received opinion in intellectual circles.

Despite the reticence of most scientists on the subject of good and evil, the scientific study of morality and human happiness is well underway.
This research is bound to bring science into conflict with religious orthodoxy and popular opinion --- just as our growing understanding of evolution has --- because the divide between facts and values is illusory in at least three senses :

\begin{itemize}

    \item
          Whatever can be known about maximizing the well-being of conscious creatures --- which is, I will argue, the only thing we can reasonably value --- must at some point translate into facts about brains and their interaction with the world at large;
    \item
          The very idea of ``objective'' knowledge (i.e., knowledge acquired through honest and reasoning) has values built into it, as every effort we make to discuss facts depends upon principles that we must first value (i.e. logical consistency, reliance on evidence, parsimony, etc.);
    \item
          Beliefs about facts and beliefs about values seem to arise from similar processes at the level of the brain:
          it appears that we have a common system for judging truth and falsity in both domains.

\end{itemize}


I will discuss each of these points in greater detail below.
Both in terms of what there is to know about the world and the brain mechanisms that allow us to know it, we will see that a clear boundary between facts and values simply does not exist.

Many readers might wonder how can we base our values on something as difficult to define as ``well-being''?
It seems to me, however, that the concept of well-being is like the concept of physical health :
it resists precise definition, and yet it is indispensable.
In fact, the meanings of both terms seem likely to remain perpetually open to revision as we make progress in science.
Today, a person can consider himself physically healthy if he is free of detectable disease, able to exercise, and destined to live into his eighties without suffering obvious decrepitude.
But this standard may change.
If the biogerontologist Aubrey de Grey is correct in viewing aging as an engineering problem that admits of a full solution, being able to walk a mile on your hundredth birthday will not always constitute ``health.''
There may come a time when not being able to run a marathon at age five hundred will be considered a profound disability.
Such a radical transformation of our view of human health would not suggest that current notions of health and sickness are arbitrary, merely subjective, or culturally constructed.
Indeed, the difference between a healthy person and a dead one is about as clear and consequential a distinction as we ever make in science.
The differences between the heights of human fulfillment and the depths of human misery are no less clear, even if new frontiers await us in both directions.

If we define ``good'' as that which supports well-being, as I will argue we must, the regress initiated by Moore’s ``open question argument'' really does stop.
While I agree with Moore that it is reasonable to wonder whether maximizing pleasure in any given instance is ``good,'' it makes no sense at all to ask whether maximizing well-being is ``good.''
It seems clear that what we are really asking when we wonder whether a certain state of pleasure is ``good,'' is whether it is conducive to, or obstructive of, some deeper form of well-being.
This question is perfectly coherent;
it surely has an answer (whether or not we are in a position to answer it);
and yet, it keeps notions of goodness anchored to the experience of sentient beings.

Defining goodness in this way does not resolve all questions of value;
it merely directs our attention to what values actually are --- the set of attitudes, choices, and behaviors that potentially affect our well-being, as well as that of other conscious minds.
While this leaves the question of what constitutes well-being genuinely open, there is every reason to think that this question has a finite range of answers.
Given that change in the well-being of conscious creatures is bound to be a product of natural laws, we must expect that this space of possibilities --- the moral landscape --- will increasingly be illuminated by science.

It is important to emphasize that a scientific account of human values --- i.e., one that places them squarely within the web of influences that link states of the world and states of the human brain --- is not the same as an evolutionary account.
Most of what constitutes human well-being at this moment escapes any narrow Darwinian calculus.
While the possibilities of human experience must be realized in the brains that evolution has built for us, our brains were not designed with a view to our ultimate fulfillment.
Evolution could never have foreseen the wisdom or necessity of creating stable democracies, mitigating climate change, saving other species from extinction, containing the spread of nuclear weapons, or of doing much else that is now crucial to our happiness in this century.

As the psychologist Steven Pinker has observed, if conforming to the dictates of evolution were the foundation of subjective well-being, most men would discover no higher calling in life than to make daily contributions to their local sperm bank.
After all, from the perspective of a man’s genes, there could be nothing more fulfilling than spawning thousands of children without incurring any associated costs or responsibilities.
But our minds do not merely conform to the logic of natural selection.
In fact, anyone who wears eyeglasses or uses sunscreen has confessed his disinclination to live the life that his genes have made for him.
While we have inherited a multitude of yearnings that probably helped our ancestors survive and reproduce in small bands of hunter-gatherers, much of our inner life is frankly incompatible with our finding happiness in today’s world.
The temptation to start each day with several glazed donuts and to end it with an extramarital affair might be difficult for some people to resist, for reasons that are easily understood in evolutionary terms, but there are surely better ways to maximize one’s long-term well-being.
I hope it is clear that the view of ``good'' and ``bad'' I am advocating, while fully constrained by our current biology (as well as by its future possibilities), cannot be directly reduced to instinctual drives and evolutionary imperatives.
As with mathematics, science, art, and almost everything else that interests us, our modern concerns about meaning and morality have flown the perch built by evolution.

\subsection{The Importance of Belief}

The human brain is an engine of belief. 
Our minds continually consume, produce, and attempt to integrate ideas about ourselves and the world that purport to be true : 
\textit{Iran is developing nuclear weapons;} 
\textit{the seasonal flu can be spread through casual contact;} 
\textit{I actually look better with gray hair.} 
What must we do to believe such statements? 
What, in other words, must a brain do to accept such propositions as true? 
This question marks the intersection of many fields : 
psychology, neuroscience, philosophy, economics, political science, and even jurisprudence. 

Belief also bridges the gap between facts and values. 
We form beliefs about facts : 
and belief in this sense constitutes most of what we know about the world --- through science, history, journalism, etc. 
But we also form beliefs about values : 
judgments about morality, meaning, personal goals, and life’s larger purpose. 
While they might differ in certain respects, beliefs in these two domains share very important features. 
Both types of belief make tacit claims about right and wrong : 
claims not merely about how we think and behave, but about how we should think and behave. 
Factual beliefs like ``water is two parts hydrogen and one part oxygen'' and ethical beliefs like ``cruelty is wrong'' are not expressions of mere preference. 
To really believe either proposition is also to believe that you have accepted it for legitimate reasons. 
It is, therefore, to believe that you are in compliance with certain norms --- that you are sane, rational, not lying to yourself, not confused, not overly biased, etc. 
When we believe that something is factually true or morally good, we also believe that another person, similarly placed, should share our belief. 
This seems unlikely to change. 
In chapter 3, we will see that both the logical and neurological properties of belief further suggest that the divide between facts and values is illusory. 

\subsection{The Bad Life and the Good Life}

For my argument about the moral landscape to hold, I think one need only grant two points :

\begin{itemize}
    \item Some people have better lives than others, and
    \item These differences relate, in some lawful and not entirely arbitrary ways, to states of the human brain and to states of the world.
\end{itemize}

To make these premises less abstract, consider two generic lives that lie somwhere near the extremes on this continuum :

\subsubsection{The Bad Life}

You are a yound widow who has lived her entire life in the midst of civil war.
Today, your seven-year-old daughter was raped and dismembered before your eyes.
Worse still, the perpetrator was your fourteen-year-old son, who was goaded to this evil at the point of a machete by a press gang of drug-addicted soldiers.
You are now running barefoot through the jungle with killers in pursuit.
While this is the worst day of your life, it is not entirely out of character with the other days of your life :
since the moment you were born, your world has been a theater of cruelty and violence.
You have never learned to read, taken a hot shower, or traveled beyond the green hell of the jungle.
Even the luckiest people you have known have experienced little more than an occasional respite from chronic hunger, fear, apathy, and confusion.
Unfortunately, you've been very unlucky, even by these bleak standards.
Your life has been one long emergency, now it is nearly over.

\subsubsection{The Good Life}

You are married to the most loving, intelligent, and charismatic person you have ever met.
Both of you have careers that are intellectually stimulating, and financially rewarding.
For decades, your wealth and social connections have allowed you to devote yourself to activites that bring you immense personal satisfaction.
One of your greatest sources of happiness has been to find creative ways to help people who have not had your good fortune in life.
In fact, you have just won a billion-dollar grant to benefit children in the developing world.
If asked, you would say that you could not imagine how your time on Earth could be better spent.
Due to a combination of good genes and optimal circumstances, you and your closest friends and family will live very long, healthy lives, untouched by crimes, sudden bereavements, and other misfortunes.

The examples I have picked, while generic, are nonetheless real --- in that they represent lives that some human beings are likely to be leading at this moment. 
While there are surely ways in which this spectrum of suffering and happiness might be extended, I think these cases indicate the general range of experience that is accessible, in principle, to most of us. 
I also think it is indisputable that most of what we do with our lives is predicated on there being nothing more important, at least for ourselves and for those closest to us, than the difference between the Bad Life and the Good Life.

Let me simply concede that if you don't see a distinction between these two lives that is worth valuing (premise 1 above), there may be nothing I can say that will attract you to my view of the moral landscape.
Likewise, if you admit that these lives are different, and that one is surely better than the other, but you believe these differences have no lawful relationship to human behavior, societal conditions, or states of the brain (premise 2), then you will also fail to see the point of my argument.
While I don't see how either premise 1 or 2 can be reasonably doubted, my experience discussing these issues suggests that I should address such skepticism, however far-fetched it may seem.

There are actually people who claim to be unimpressed by the difference between the Bad Life and the Good Life. 
I have even met people who will go so far as to deny that any difference exists. 
While they will acknowledge that we habitually speak and act as if there were a continuum of experience that can be described by words like ``misery,'' ``terror,'' ``agony,'' ``madness,'' etc., on one end and ``well-being,'' ``happiness,'' ``peace,'' ``bliss,'' etc., on the other, when the conversation turns to philosophical and scientific matters, such people will say learned things like, ``but, of course, that is just how we play our particular language game. It doesn’t mean there is a difference in reality.'' 
One hopes that these people take life’s difficulties in stride. 
They also use words like ``love'' and ``happiness,'' from time to time, but we should wonder what these terms could signify that does not entail a preference for the Good Life over the Bad Life. 
Anyone who claims to see no difference between these two states of being (and their concomitant worlds), should be just as likely to consign himself and those he ``loves'' to one or the other at random and call the result ``happiness.'' 

Ask yourself, if the difference between the Bad Life and the Good Life doesn’t matter to a person, what could possibly matter to him? 
Is it conceivable that something might matter more than this difference, expressed on the widest possible scale? 
What would we think of a person who said, ``Well, I could have delivered all seven billion of us into the Good Life, but I had other priorities.'' 
Would it be \textit{possible} to have other priorities? 
Wouldn’t any real priority be best served amid the freedom and opportunity afforded by the Good Life? 
Even if you happen to be a masochist who fancies an occasional taunting with a machete, wouldn’t this desire be best satisfied in the context of the Good Life? 

Imagine someone who spends all his energy trying to move as many people as possible toward the Bad Life, while another person is equally committed to undoing this damage and moving people in the opposite direction: 
Is it conceivable that you or anyone you know could overlook the differences between these two projects? 
Is there any possibility of confusing them or their underlying motivations? 
And won’t there necessarily be objective conditions for these differences? 
If, for instance, one’s goal were to place a whole population securely in the Good Life, wouldn’t there be more and less effective ways of doing this? 
How would forcing boys to rape and murder their female relatives fit into the picture? 

I do not mean to belabor the point, but the point is crucial --- and there is a pervasive assumption among educated people that either such differences don’t exist, or that they are too variable, complex, or culturally idiosyncratic to admit of general value judgments. 
However, the moment one grants there is a difference between the Bad Life and the Good Life that lawfully relates to states of the human brain, to human behavior, and to states of the world, one has admitted that there are right and wrong answers to questions of morality. 
To make sure this point is nailed down, permit me to consider a few more objections :  

\textit{What if, seen in some larger context, the Bad Life is actually better than the Good Life --- e.g., what if all those child soldiers will be happier in some afterlife, because they have been purified of sin or have learned to call God by the right name, while the people in the Good Life will get tortured in some physical hell for eternity?} 

If the universe is really organized this way, much of what I believe will stand corrected on the Day of Judgment. 
However, my basic claim about the connection between facts and values would remain unchallenged. 
The rewards and punishments of an afterlife would simply alter the temporal characteristics of the moral landscape. 
If the Bad Life is actually better over the long run than the Good Life --- because it wins you endless happiness, while the Good Life represents a mere dollop of pleasure presaging an eternity of suffering --- then the Bad Life would surely be better than the Good Life. 
If this were the way the universe worked, we would be morally obligated to engineer an appropriately pious Bad Life for as many people as possible. 
Under such a scheme, there would still be right and wrong answers to questions of morality, and these would still be assessed according to the experience of conscious beings. 
The only thing left to be decided is how reasonable it is to worry that the universe might be structured in so bizarre a way. 
It is not reasonable at all, I think --- but that is a different discussion. 

\textit{What if certain people would actually prefer the Bad Life to the Good Life? 
Perhaps there are psychopaths and sadists who can expect to thrive in the context of the Bad Life and would enjoy nothing more than killing other people with machetes.} 

Worries like this merely raise the question of how we should value dissenting opinions. 
Jeffrey Dahmer’s idea of a life well lived was to kill young men, have sex with their corpses, dismember them, and keep their body parts as souvenirs. 
We will confront the problem of psychopathy in greater detail in chapter 3. 
For the moment, it seems sufficient to notice that in any domain of knowledge, we are free to say that certain opinions do not count. 
In fact, we must say this for knowledge or expertise to count at all. 
Why should it be any different on the subject of human well-being?  

Anyone who doesn’t see that the Good Life is preferable to the Bad Life is unlikely to have anything to contribute to a discussion about human well-being. 
Must we really argue that beneficence, trust, creativity, etc., enjoyed in the context of a prosperous civil society are better than the horrors of civil war endured in a steaming jungle filled with aggressive insects carrying dangerous pathogens? 
I don’t think so. 
In the next chapter, I will argue that anyone who would seriously maintain that the opposite is the case --- or even that it \textit{might} be the case --- is either misusing words or not taking the time to consider the details. 

If we were to discover a new tribe in the Amazon tomorrow, there is not a scientist alive who would assume \textit{a priori} that these people must enjoy optimal physical health and material prosperity. 
Rather, we would ask questions about this tribe’s average lifespan, daily calorie intake, the percentage of women dying in childbirth, the prevalence of infectious disease, the presence of material culture, etc. 
Such questions would have answers, and they would likely reveal that life in the Stone Age entails a few compromises. 
And yet news that these jolly people enjoy sacrificing their firstborn children to imaginary gods would prompt many (even most) anthropologists to say that this tribe was in possession of an alternate moral code every bit as valid and impervious to refutation as our own. 
However, the moment one draws the link between morality and well-being, one sees that this is tantamount to saying that the members of this tribe must be as fulfilled, psychologically and socially, as any people on earth. 
The disparity between how we think about physical health and mental/societal health reveals a bizarre double standard : 
one that is predicated on our not knowing --- or, rather, on our \textit{pretending} not to know --- anything at all about human well-being. 

Of course, some anthropologists have refused to follow their colleagues over the cliff. 
Robert Edgerton performed a book-length exorcism on the myth of the ``noble savage,'' detailing the ways in which the most influential anthropologists of the 1920s and 1930s --- such as Franz Boas, Margaret Mead, and Ruth Benedict --- systematically exaggerated the harmony of folk societies and ignored their all too frequent barbarism or reflexively attributed it to the malign influence of colonialists, traders, missionaries, and the like. 
Edgerton details how this romance with mere difference set the course for the entire field. 
Thereafter, to compare societies in moral terms was deemed impossible. 
Rather, it was believed that one could only hope to understand and accept a culture on its own terms. 
Such cultural relativism became so entrenched that by 1939 one prominent Harvard anthropologist wrote that this suspension of judgment was ``probably the most meaningful contribution which anthropological studies have made to general knowledge.'' 
Let’s hope not. 
In any case, it is a contribution from which we are still struggling to awaken. 

Many social scientists incorrectly believe that all long-standing human practices must be evolutionarily adaptiven : 
for how else could they persist? 
Thus, even the most bizarre and unproductive behaviors --- female genital excision, blood feuds, infanticide, the torture of animals, scarification, foot binding, cannibalism, ceremonial rape, human sacrifice, dangerous male initiations, restricting the diet of pregnant and lactating mothers, slavery, potlatch, the killing of the elderly, sati, irrational dietary and agricultural taboos attended by chronic hunger and malnourishment, the use of heavy metals to treat illness, etc. --- have been rationalized, or even idealized, in the fire-lit scribblings of one or another dazzled ethnographer. 
But the mere endurance of a belief system or custom does not suggest that it is adaptive, much less wise. 
It merely suggests that it hasn’t led directly to a society’s collapse or killed its practitioners outright. 

The obvious difference between genes and \textit{memes} (e.g., beliefs, ideas, cultural practices) is also important to keep in view. 
The latter are \textit{communicated}; 
they do not travel with the gametes of their human hosts. 
The survival of memes, therefore, is not dependent on their conferring some actual benefit (reproductive or otherwise) on individuals or groups. 
It is quite possible for people to traffic in ideas and other cultural products that diminish their well-being for centuries on end. 

Clearly, people can adopt a form of life that needlessly undermines their physical health --- as the average lifespan in many primitive societies is scarcely a third of what it has been in the developed world since the middle of the twentieth century. 
Why isn’t it equally obvious that an ignorant and isolated people might undermine their psychological well-being or that their social institutions could become engines of pointless cruelty, despair, and superstition? 
Why is it even slightly controversial to imagine that some tribe or society could harbor beliefs about reality that are not only false but demonstrably harmful? 

Every society that has ever existed has had to channel and subdue certain aspects of human nature --- envy, territorial violence, avarice, deceit, laziness, cheating, etc. --- through social mechanisms and institutions. 
It would be a miracle if all societies --- irrespective of size, geographical location, their place in history, or the genomes of their members --- had done this equally well. 
And yet the prevailing bias of cultural relativism assumes that such a miracle has occurred not just once, but always. 
Let’s take a moment to get our bearings. 
From a factual point of view, is it possible for a person to believe the wrong things? 
Yes. 
It is possible for a person to value the wrong things (that is, to believe the wrong things about human well-being)? 
I am arguing that the answer to this question is an equally emphatic ``yes'' and, therefore, that science should increasingly inform our values. 
Is it possible that certain people are incapable of wanting what they should want? 
Of course --- just as there will always be people who are unable to grasp specific facts or believe certain true propositions. 
As with every other description of a mental capacity or incapacity, these are ultimately statements about the human brain. 

\newpage
\section{Chapter 1 : Moral Truths}

\newpage
\section{Chapter 2 : Good And Evil}

\newpage
\section{Chapter 3 : Belief}

\newpage
\section{Chapter 4 : Religion}

\newpage
\section{Chapter 5 : The Future of Happiness}

\newpage
\section{Acknowledgements}

\newpage
\section{Notes}

\newpage
\section{References}

\newpage
\section{Index}

\end{document}
