\documentclass[a4paper,12pt]{extbook}
\usepackage[utf8]{inputenc}

\title{A Review of the Moral Landscape}
\author{Hanif B Ariffin}

\begin{document}

\maketitle
\tableofcontents

\newpage
\section{Introduction}

What I think as a proper introduction to the book :

``While the argument I make in this book is bound to be controversial, it rests on a very simple premise:
human well-being entirely depends on events in the world and on states of the human brain.
Consequently, there must be scientific truths to be known about it.''

Structure of the Book :

\begin{itemize}
    \item Brief introduction
    \item We Live in the Physical World
    \item The Moral Landscape
    \item An Analogy to Health
\end{itemize}

\newpage
\section{Universal Moral Truths}

The following exact quote from the book,

``
Many people seem to think that a universal conception of morality requires that we find moral principles that admit of no exceptions.
If, for instance, it is truly wrong to lie, it must always be wrong to lie—and if one can find a single exception, any notion of moral truth must be abandoned.
But the existence of moral truth—that is, the connection between how we think and behave and our well-being—does not require that we define morality in terms of unvarying moral precepts.
Morality could be a lot like chess:
there are surely principles that generally apply, but they might admit of important exceptions.
If you want to play good chess, a principle like “Don’t lose your Queen” is almost always worth following.
But it admits of exceptions:
sometimes sacrificing your Queen is a brilliant thing to do;
occasionally, it is the only thing you can do.
It remains a fact, however, that from any position in a game of chess there will be a range of objectively good moves and objectively bad ones.
If there are objective truths to be known about human well-being—if kindness, for instance, is generally more conducive to happiness than cruelty is—then science should one day be able to make very precise claims about which of our behaviors and uses of attention are morally good, which are neutral, and which are worth abandoning.
''

Can be summarised the following fact :
Our conception of morality must be bound to the physical reality of its part : the beings involved in experiencing it.

Thus, it is foolish to assume that there exist 1 universal moral truth, because such claims are void of any bearing on the physical reality it claims to answer.

\newpage
\section{Criticisms}

I will now lay down my criticism of the book.

\subsection{Proof of Common Knowledge}

In the book, the author said that,

```
Consider \dots the connection between early childhood experience, emotional bonding, and a person’s ability to form healthy relationships later in life.
We know, of course, that emotional neglect and abuse are not good for us, psychologically or socially
'''

There were many such examples, where the author seeks the reader to conjure common sense to support his arguments.

\subsection{The Local Maxima}

Consider the Moral Landscape, I argue that there is discussion to be had regarding change.
In this criticism, lend me 2 assumptions :

\begin{itemize}
    \item A religious life is a local maxima
    \item Changes are incremental
\end{itemize}

By definition, the coordinates surrounding a local maxima must have lower values.
Additionally, the next neighboring local maxima, which may or may not have higher peaks, will have some distance away.
This distance must therefore be traversed incrementally, incurring a cost because we are now journeying down the moral hill.

Several questions will rise from this:

\begin{itemize}
    \item Is it moral to perform this journey?
    \item Assuming that there are many moral peaks, each higher than the other, are we bound to eternally traverse these valleys?
\end{itemize}

\end{document}
