\documentclass[a4paper,10pt]{article}
\usepackage[utf8]{inputenc}

%opening
\title{The Honest Thief : \textit{From the Memoirs of an Unknown}}
\author{Fyodor Dostoevsky}

\begin{document}

\maketitle

One morning as I was leaving for my office, Agrafena, my cook, washerwoman, and housekeeper, came into my room and, to my great surprise, began a conversation with me.
Until that morning this simple, ordinary woman of the people had been so uncommunicative that, except for a few words about my dinner each day, she had for the last six years scarcely uttered a word to me.
At least I never heard her speak of anything else.

``I'd like to have a word with you, sir,'' she began abruptly.
``Why don't you let the little room?''

``Which little room?''

``Why, as if you didn't know, sir.
The one next to the kitchen, of course!''

``What for?''

``What for, sir?
Why, surely sir, you know who would take it.
A lodger, of course.''

``But, my good woman, who'd like to live in a cubbyhole like that?
Why, it's nothing but a boxroom.
I doubt if you could put a bed in it, and even if you could, there wouldn't be any room left to move about it.''

``Why, sir, nobody wants to live in it.
All he wants is a place to sleep in.
He'd live on the windowsill.''

``Which windowsill?''

``Why, as if you didn't know, sir!
The windowsill in the passage, of course.
He could sit there and sew, or do whatever he liked.
He could sit on a chair, if he liked.
He's got a chair, and a table, too.
He's got everything, sir.''

``But who is he?''

``Oh, he's a good man, sir.
He's had a lot of experience in his life, he has, sir.
I'll cook for him, and I'll only charge him ten roubles a month for his board and lodging.''

After the exercise of a great deal of patience, I found that an elderly man had persuaded or somehow induced Agrafena to admit him to her kitchen as apaying guest.
Now, I knew very well that if Agrafena ever took it in her head to do a thing, it had better be done at once, or she would give me no peace.
For whenever anything was not to her liking, Agrafena became moddy and fell a prey to the blackest melancholy which lasted for a fortnight or three weeks.
During that time my dinners were uneatable, my floors remained unscrubbed, and several indispensable articles were missing from my personal, in short, my life became one long chapter of the most unfortunate accidents.

\end{document}
