\documentclass[a4paper, 12pt]{extbook}
\usepackage[utf8]{inputenc}
\usepackage[a4paper, total={7in, 10in}]{geometry}
\usepackage{listings}
\usepackage{hyperref}
% Table stuff
\usepackage{tabularx,ragged2e,booktabs,caption}
\newcolumntype{C}[1]{>{\Centering}m{#1}}
\renewcommand\tabularxcolumn[1]{C{#1}}
% Extension for subsubsub...section of depth>3
\usepackage{titlesec}

\titleclass{\subsubsubsection}{straight}[\subsection]

\newcounter{subsubsubsection}[subsubsection]
\renewcommand\thesubsubsubsection{\thesubsubsection.\arabic{subsubsubsection}}
\renewcommand\theparagraph{\thesubsubsubsection.\arabic{paragraph}} % optional; useful if paragraphs are to be numbered

\titleformat{\subsubsubsection}
  {\normalfont\normalsize\bfseries}{\thesubsubsubsection}{1em}{}
\titlespacing*{\subsubsubsection}
{0pt}{3.25ex plus 1ex minus .2ex}{1.5ex plus .2ex}

\makeatletter
\renewcommand\paragraph{\@startsection{paragraph}{5}{\z@}%
  {3.25ex \@plus1ex \@minus.2ex}%
  {-1em}%
  {\normalfont\normalsize\bfseries}}
\renewcommand\subparagraph{\@startsection{subparagraph}{6}{\parindent}%
  {3.25ex \@plus1ex \@minus .2ex}%
  {-1em}%
  {\normalfont\normalsize\bfseries}}
\def\toclevel@subsubsubsection{4}
\def\toclevel@paragraph{5}
\def\toclevel@paragraph{6}
\def\l@subsubsubsection{\@dottedtocline{4}{7em}{4em}}
\def\l@paragraph{\@dottedtocline{5}{10em}{5em}}
\def\l@subparagraph{\@dottedtocline{6}{14em}{6em}}
\makeatother

\setcounter{secnumdepth}{4}
\setcounter{tocdepth}{4}

\begin{document}

\section{Introduction}

C++ is an extremely powerful language that allows you to write efficient program without having to get dirty with low level construct.
This is possible because C++ design puts a lot of emphasis on having zero-cost abstractions.

\section{C++ Basics}

We will now go through the basics of C++ that you will commonly encounter.

\subsection{Variable Assignment and Initialization}

\subsubsection{Variable Assignment}

After a variable has been defined, you can give it values using the \verb|=|.
This is called a \verb|copy assignemnt|.

\begin{lstlisting}[language=c++]
// defining a variable
int width;
// assigning it with a value of 10
width = 10;
// you can immediately assign it values, of course!
int height = 100;
\end{lstlisting}

\end{document}
