\documentclass{article}

\title{Why Am I Like This?}
\author{Anonymous}

\begin{document}

\maketitle

\begin{abstract}
    Knowing oneself is a hard and painful activity.
    Although we are aware of our flaws and shortcomings, we tend to ignore them.
    As a result, the failure broods and at some point it will consume us.
    Because of this, it is perhaps instructive to write about oneself.
    
    Some people have said that writing is the only way one can explore one's mind.
    Writing is also very helpful to thinking, because it is thinking made into being.
    Thus, some argue, it is easier to be perceived.
\end{abstract}

\section{When Did I Start To Write?}

I think I had always been interested in literature.
I like the fantastical and imaginary stories that many great authors can compose.
Sometimes when I am fully indulge in the text, my mind will wandef off from terrestrial reality into the wonderful other-world.

In school, we had English lessons.
Many students despise this class, particularly because, well, it's English.
Its bothersome enough to learn one's mothers tongue, and now we have to learn anothers?

For some reason however, I am very good it.
Well, I think I was, compared to my peers.
I had a special knack for writing short stories essay.
We often had multiple type of essays to choose from.
I would always take the short story.

Despite all of this, I never quite get literature.
To me, it is just that, a story.
A story to be told, listened, imagined, then forgotten.
I never quite understood how great literature are great.
I thought, and now only merely feel, that great literature are just great stories.

\section{I Am A Sick Man}

The short story \textit{Notes From Underground} by Fyodor Dostoyevsky had a great line along the lines of :

"I am a sick man, I am a despicable man...."

Perhaps this is a bit pretentious, but I think I am a sick man.

\section{The Future}


\end{document}
