\documentclass[a4paper,10pt]{book}
\usepackage[utf8]{inputenc}

\title{Notes From OpenGL Redbook}
\author{Hanif Bin Ariffin}

\begin{document}

\maketitle

\section{Chapter 1 : Introduction to OpenGL}

\subsection{What Is OpenGL}

This chapter describes the purpose of OpenGL, what it can and cannot do in creating computer-generated graphics.
In particular, OpenGL's purpose is very specific in that it does not provide any APIs for creating windows, what type of representation you have to use for models and many other functions one might associate with making a program.

The major operations that an OpenGL application would perform is the following:

\begin{itemize}
 \item Specify the data for constructing shapres from OpenGL's geometric primitives
 \item Execute various shaders to perform calculations on the input primitives to determine their position,color, and other rendering attributes.
 \item Finally, execute a fragment shader for each of the fragments generated by rasterization, which will determine the fragment's final color and position.
 \item Possibly perform additional per-fragment operations such as determining if the object that the fragment was generated from is visible, or blending the fragment's color with the current color in that screen location.
\end{itemize}

\subsection{First Look at an OpenGL Program}

A concept that is essential to using OpenGL is shaders, which could be think of as a basic verison of a C program that compiles to a something that is can be understood by GPUs.

In OpenGL, there are 4 shader stages that you can use.
The most common are vertex shaders (which process vertex data) and fragment shaders (which operate on the fragments generated by the rasterizer).
Both vertex and fragment shders are required in every OpenGL program!

The final generated image consists of pixels drawn on the screen;
a pixel is the smallest visible element on your display.
The pixels in your system are stored in a framebuffer, which is a chunk of memory that the graphics hardware manages, and feeds to your display device.

At some point, we have to do something that is called \textit{shader plumbing} where we associate the data in your application with variables in \textit{shader programs}.
This is discussed in Chapter 2.

\subsection{OpenGL Syntax}

This chapter talks about the common naming conventions used by OpenGL.
It also talks a lot about the integral types used by OpenGL.
I wonder how important this actually is?
Have anyone broken ports because the types mismatched?

\end{document}
